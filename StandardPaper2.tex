%\usepackage{etoolbox}
%\makeatletter
%\let\ams@starttoc\@starttoc
%\makeatother
%\makeatletter
%\let\@starttoc\ams@starttoc
%\patchcmd{\@starttoc}{\makeatletter}{\makeatletter\parskip\z@}{}{}
%\makeatother

%\usepackage[parfill]{parskip}

\usepackage[colorlinks=true,linkcolor=blue,citecolor=blue,urlcolor=blue]{hyperref}
\usepackage{bookmark}
\usepackage{amsthm,thmtools,amssymb,amsmath,amscd}

%\usepackage[bibstyle=alphabetic,citestyle=alphabetic,backend=bibtex]{biblatex}
%\bibliography{Bibliography}

\usepackage{fancyhdr}
\usepackage{esint}

\usepackage{enumerate}

\usepackage{pictexwd,dcpic}

\usepackage{graphicx}

\swapnumbers
\declaretheorem[name=Theorem,numberwithin=section]{theorem}
\declaretheorem[name=Remark,style=remark,sibling=theorem]{remark}
\declaretheorem[name=Lemma,sibling=theorem]{lemma}
\declaretheorem[name=Proposition,sibling=theorem]{prop}
\declaretheorem[name=Definition,style=definition,sibling=theorem]{defn}
\declaretheorem[name=Corollary,sibling=theorem]{cor}
\declaretheorem[name=Assumption,style=remark,sibling=theorem]{ass}
\declaretheorem[name=Example,style=remark,sibling=theorem]{example}

\numberwithin{equation}{section}

\usepackage{cleveref}
\crefname{lemma}{Lemma}{Lemmata}
\crefname{prop}{Proposition}{Propositions}
\crefname{thm}{Theorem}{Theorems}
\crefname{cor}{Corollary}{Corollaries}
\crefname{defn}{Definition}{Definitions}
\crefname{example}{Example}{Examples}
\crefname{rem}{Remark}{Remarks}
\crefname{ass}{Assumption}{Assumptions}
\crefname{not}{Notation}{Notation}

%Symbols
\newcommand{\NN}{\mathbb{N}}
\newcommand{\RR}{\mathbb{R}}
\newcommand{\ZZ}{\mathbb{Z}}
\renewcommand{\SS}{\mathbb{S}}
\newcommand{\HH}{\mathbb{H}}
\newcommand{\CC}{\mathbb{C}}
\newcommand{\KK}{\mathbb{K}}

%Mathematical operators
\newcommand{\ip}[2]{\left\langle{#1},{#2}\right\rangle}
\newcommand{\abs}[1]{\left\lvert#1\right\rvert}
\DeclareMathOperator{\hess}{Hess}
\DeclareMathOperator{\sym}{Sym}
\DeclareMathOperator{\Tr}{Tr}
\DeclareMathOperator{\dive}{div}
\DeclareMathOperator{\id}{id}
\DeclareMathOperator{\pr}{pr}
\DeclareMathOperator{\Diff}{Diff}
\DeclareMathOperator{\supp}{supp}
\DeclareMathOperator{\graph}{graph}
\DeclareMathOperator{\osc}{osc}
\DeclareMathOperator{\const}{const}
\DeclareMathOperator{\dist}{dist}
\DeclareMathOperator{\loc}{loc}
\DeclareMathOperator{\grad}{grad}
\DeclareMathOperator{\ric}{Ric}
\DeclareMathOperator{\Rm}{Rm}
\DeclareMathOperator{\sff}{A}
\DeclareMathOperator{\weingarten}{\mathcal{W}}
\DeclareMathOperator{\inj}{inj}

%\parindent 0 pt

\protected\def\ignorethis#1\endignorethis{}
\let\endignorethis\relax
\def\TOCstop{\addtocontents{toc}{\ignorethis}}
\def\TOCstart{\addtocontents{toc}{\endignorethis}}

% Local Definitions
\usepackage{xparse}
\DeclareDocumentCommand{\bgg}{ O{\cdot} O{\cdot} }{\ip{{#1}}{{#2}}}
\DeclareMathOperator{\nor}{\nu}
\DeclareMathOperator{\bgD}{\widehat{D}}
\DeclareMathOperator{\bgRm}{\widehat{Rm}}
\DeclareMathOperator{\sig}{\sigma}

\DeclareMathOperator{\g}{g}
\DeclareMathOperator{\D}{\nabla}

\DeclareMathOperator{\vel}{\xi}
\DeclareMathOperator{\tangvel}{\zeta}
\DeclareMathOperator{\speed}{f}

\newcommand{\aff}[1]{\overline{#1}}
\DeclareMathOperator{\affshape}{\aff{\weingarten}}
\DeclareMathOperator{\affg}{\aff{\g}}
\DeclareMathOperator{\affgrad}{\aff{\grad}}
\DeclareMathOperator{\affsff}{\aff{\sff}}

\DeclareMathOperator{\diff}{K}
