\documentclass{amsart}
\usepackage[ocgcolorlinks,linktoc=all]{hyperref}
\usepackage{cancel}
\hypersetup{citecolor=blue,linkcolor=red}
\newtheorem{theorem}{Theorem}
\newtheorem*{thmA}{Theorem}
\newtheorem*{thmB}{Theorem}
\newtheorem*{rem}{Remark}
\newtheorem*{thmmain}{Theorem}
\newtheorem{lemma}[theorem]{Lemma}
\newtheorem{proposition}[theorem]{Proposition}
\newtheorem*{propmain}{Proposition}
\newtheorem{corollary}[theorem]{Corollary}
\theoremstyle{definition}
\newtheorem{definition}[theorem]{Definition}
\newtheorem{example}[theorem]{Example}
\newtheorem{xca}[theorem]{Exercise}

\theoremstyle{remark}
\newtheorem{remark}[theorem]{Remark}

\newcommand{\abs}[1]{\lvert#1\rvert}
\numberwithin{equation}{section}

\newcommand{\blankbox}[2]{%
  \parbox{\columnwidth}{\centering
    \setlength{\fboxsep}{0pt}%
    \fbox{\raisebox{0pt}[#2]{\hspace{#1}}}%
  }%
}

\begin{document}

\title[]
 {}

\curraddr{}
\email{}
\date{\today}

\dedicatory{}
\subjclass[2010]{}
\keywords{}

\begin{abstract}

\end{abstract}

\maketitle
Let $M^n$ be an $n$-dimensional smooth submanifold of $(N^{n+1},\hat{g})$ given by an embedding $F_0.$ A one-parameter family of immersions $F\colon M^n\times [0,T)\to (N^{n+1},\hat{g})$ is said to be a solution of the $ p $-GCF if
\begin{align}
\partial_tF=\frac{p}{|p|}\xi,\quad F(\cdot,0)=F_0(\cdot).
\end{align}
Here $\xi $ is a smooth transverse inward-pointing vector field defined by
\begin{align}
\xi:=F_{\ast}(-\operatorname{grad}_hK^p)+ K ^{ p }\nu,
\end{align}
where $ K $ is the Gauss curvature, $\nu$ is an inward unit normal vector and $\operatorname{grad}_hK^{p}$ is the gradient of $K^{p}$ with respect to the second fundamental $h$ induced by $\nu,$ namely
\[
h(X, \operatorname{grad}_h K^p) = X(K^p)
\]
for all vectors $X$ tangent to $M$.

We make use of the theory of affine immersions as described in \cite{MR1311248}. In what follows for simplicity, we set $ f = K ^{ p }$ and $W:=-\operatorname{grad}_hK^{p} = - \operatorname{grad}_h f.$

Let $\bar{g}$ be the induced metric by the immersion. The Gauss formula is given by
\begin{align}\label{gauss equ}
\hat{\nabla}_YF_{\ast}(X)=F_{\ast}(\bar{\nabla}_YX)+h(X,Y)\nu.
\end{align}
Define the non-degenerate metric
\begin{align}
g(X,Y):=\frac{h(X,Y)}{ f }.
\end{align}
The second fundamental form with respect to $\xi$ is defined by
\begin{align}
\hat{\nabla}_X\xi:=-F_{\ast}(AX)
\end{align}
for $X\in TM$. Note that
\begin{align*}
\hat{\nabla}_X \xi &= \hat{\nabla}_X (F_{\ast}(W) + f \nu) \\
&= \hat{\nabla}_X F_{\ast}(W) + X(f)\nu + f \hat{\nabla}_X \nu \\
&= F_{\ast}(\bar{\nabla}_X W) - h(X, \operatorname{grad}_h f) \nu + X(f) \nu + f \hat{\nabla}_X \nu \\
&= F_{\ast} (\bar{\nabla}_X W) + f \hat{\nabla}_X \nu,
\end{align*}
which is tangential since $\nu$ has unit length. Thus,
\[
AX = \bar{\nabla}_X W + f F_{\ast}^{-1} \hat{\nabla}_X \nu.
\]

The vector field $\xi$ induces a torsion free connection $\tilde{\nabla}$:
\begin{align}\label{gauss equ2}
\hat{\nabla}_YF_{\ast}(X)=F_{\ast}(\tilde{\nabla}_YX)+g(X,Y)\xi.
\end{align}
This connection is well defined (i.e. $F_{\ast}(\tilde{\nabla}_Y X)$ is tangent to $F(M)$) since for all $X,Y\in TM$ we have
\begin{align}
F_{\ast}(\tilde{\nabla}_Y X) &= F_{\ast}(\bar{\nabla}_Y X) + h(X, Y) \nu - g(X, Y) (F_{\ast}(W) + f \nu) \\
&= F_{\ast}(\bar{\nabla}_YX)-g(X,Y)F_{\ast}(W) \nonumber.
\end{align}

The connection $\tilde{\nabla}$ is not necessarily the Levi-Civita connection for $g$. Let $\nabla$ be the Levi-Civita connection for $g$; therefore, $\nabla$ is the unique torsion free connection satisfying $\nabla g=0.$

Define the difference tensor
\begin{align*}
D_YX:=\tilde{\nabla}_{Y}X-\nabla_{Y}X.
\end{align*}
Since both $\tilde{\nabla}$ and $\nabla$ are torsion free, we have $D_YX=D_XY.$

Therefore, from (\ref{gauss equ2}) we get

\begin{align}\label{gauss equ3}
\operatorname{Hess}F(X,Y):=\hat{\nabla}_YF_{\ast}(X)-F_{\ast}(\nabla_YX)=F_{\ast}(D_YX)+g(X,Y)\xi.
\end{align}

The cubic tensor is defined by
\begin{align}
C(X,Y,Z)=g(D_XY,Z).
\end{align}
From (\ref{gauss equ3}) it follows that
\begin{align}
C(X,Y,Z)
=-\frac{1}{2}\tilde{\nabla}_Xg(Y,Z).
\end{align}
Except in the case $ p =\frac{1}{n+2}$, $ C$ does not enjoy being trace free with respect to each of its two indices.


The second covariant derivatives of $ \xi$ is given by
\begin{align}
\operatorname{Hess}\xi(X,Y)&=\hat{\nabla}_Y\hat{\nabla}_X\xi-\hat{\nabla}_{\nabla_YX}\xi\\
&=-\hat{\nabla}_YF_{\ast}(AX)+F_{\ast}(A\nabla_YX)\nonumber\\
&=-F_{\ast}(\tilde{\nabla}_{Y}AX)-g(A X,Y)\xi+F_{\ast}(A\nabla_YX)\nonumber\\
&=-F_{\ast}(\nabla_{Y}AX)-F_{\ast}(D_{Y}AX)-g(A X,Y)\xi+F_{\ast}(A\nabla_YX)\nonumber\\
&=-F_{\ast}((\nabla_Y A)X)-F_{\ast}(D_{Y}AX)-g(A X,Y)\xi.\nonumber
\end{align}
On the other hand,
\begin{align}
\widehat{Rm}(X,Y,\xi)&=\hat{\nabla}^2_{X,Y}\xi-\hat{\nabla}^2_{Y,X}\xi\\
&=\operatorname{Hess}\xi(X,Y)-\operatorname{Hess}\xi(Y,X)-\hat{\nabla}_{\hat{\nabla}_YX-\hat{\nabla}_XY}\xi-\hat{\nabla}_{\nabla_XY-\nabla_YX}\xi\nonumber\\
&=\operatorname{Hess}\xi(X,Y)-\operatorname{Hess}\xi(Y,X).\nonumber
\end{align}

\begin{lemma}
\begin{align}
\partial_t\nu=0,
\end{align}
\begin{align}
\partial_t f =|p|(f\mathcal{H}+\operatorname{tr} (\widehat{Rm}(\cdot,\xi,\cdot,\nu))),
\end{align}
\begin{align}
\partial_tg_{ij}=-\frac{p}{|p|}( A _{ij}+p  \mathcal{H}g_{ij}),
\end{align}
\begin{align}
\partial_t\xi=- |p|(g^{ij}\partial_i\mathcal{H}\partial_jF- \mathcal{H}\xi),
\end{align}
%\begin{align}
%\partial_t  C _{ijm}=&- p  \mathcal{H} C _{ijm}+\frac{ p }{2}(\partial_i\mathcal{H}g_{jm}+\partial_j\mathcal{H}g_{im}+\partial_m\mathcal{H}g_{ij})\\
%&-\frac{1}{2}( A _{ij;m}- A _m^l C _{lij})-\frac{1}{2} A _i^l C _{ljm}-\frac{1}{2} A _j^l C _{lim}-\frac{1}{2} A _m^l C _{lij}.
%\end{align}
Here $\mathcal{H}$ is the mean curvature associated with $\xi;$ that is, $\mathcal{H}=g^{ij} A _{ij}=\sum  A _i^i.$
\end{lemma}
\begin{proof}
Since $\partial_t \nu$ is tangential, we need only to calculate
\begin{align*}
\partial_t\nu&=\langle \partial_t\nu,\partial_iF\rangle \bar{g}^{ij}\partial_jF\\
&=-\langle \nu,\partial^2_{ti}F\rangle \bar{g}^{ij}\partial_jF\\
&=\frac{p}{|p|}\langle \nu, A _i^k\partial_kF\rangle \bar{g}^{ij}\partial_jF=0.
\end{align*}
To calculate the evolution equation of $ f $, we need first to calculate the evolution equation of $\bar{g}_{ij}$ and $h_{ij}.$
\begin{align*}
\partial_t \bar{g}_{ij}&=\partial_t \langle \partial_iF,\partial_jF\rangle\\
&=\langle \partial_{ti}^2F,\partial_jF\rangle+\langle \partial_{tj}^2F,\partial_iF\rangle\\
&=-\frac{p}{|p|}( A _{i}^k\bar{g}_{kj}+ A _{j}^k\bar{g}_{ki}).
\end{align*}



Using $h(X,Y)=\langle \hat{\nabla}_{Y}X, \nu\rangle$ and $\partial_t\nu=0$ we calculate
\begin{align*}
\partial_t h(X,Y)&=\hat{\nabla}_{\frac{\partial}{\partial t}}\langle \hat{\nabla}_{Y}X,\nu\rangle\\
&=\frac{p}{|p|}\langle \hat{\nabla}_{\xi}\hat{\nabla}_{Y}X,\nu\rangle\\
&=\frac{p}{|p|}\langle \hat{\nabla}_{X}\hat{\nabla}_{Y}\xi,\nu\rangle+ \frac{p}{|p|}\langle \widehat{Rm}(Y,\xi)X,\nu\rangle\\
&=-\frac{p}{|p|}\langle \hat{\nabla}_{X}AY,\nu\rangle+ \frac{p}{|p|}\langle \widehat{Rm}(Y,\xi)X,\nu\rangle\\
&=\frac{p}{|p|}\langle AY,\hat{\nabla}_{X}\nu\rangle+ \frac{p}{|p|}\langle \widehat{Rm}(Y,\xi)X,\nu\rangle\\
&=-\frac{p}{|p|} f  g(A(Y),X)+\frac{p}{|p|}\langle \widehat{Rm}(Y,\xi)X,\nu\rangle.
\end{align*}
Therefore, using $\partial_t K=\frac{\partial K}{\partial h_{ij}}\partial_th_{ij}+\frac{\partial K}{\partial \bar{g}_{ij}}\partial_t\bar{g}_{ij}$ we get
\begin{align*}
\partial_tK=\frac{pK}{|p|}(\mathcal{H}+\operatorname{tr}_h (\widehat{Rm}(\cdot,\xi,\cdot,\nu)))\Rightarrow \partial_tf=|p|(f\mathcal{H}+\operatorname{tr} (\widehat{Rm}(\cdot,\xi,\cdot,\nu))).
\end{align*}
The evolution equation of $g_{ij}$ follows from the evolution equations of $ f $ and $h_{ij}.$ The evolution equation of $\xi$ follows from the evolution equations of $g_{ij}, F,  f ,\nu.$ %Next we proceed to calculate the evolution equation of $ C _{ijk}.$ To do so, in a normal coordinates with  $\partial_kg_{ij}=\Gamma_{ij}^k=0$, using the evolution equation of $g_{ij}$ we calculate the evolution equation of $\partial_t\Gamma_{ij}^j:$
%\begin{align*}
%\partial_t\Gamma_{ij}^k=-\frac{ p }{2}\left(\partial_i\mathcal{H}\delta_j^k+\partial_j\mathcal{H}\delta_i^k-g^{kl}\partial_lg_{ij}\right)-\frac{1}{2}( A ^k_{j;i}+ A ^k_{i;j}-g^{kl} A _{ij;l}).
%\end{align*}
%Thus
%\begin{align*}
%\partial_tF_{;ij}&=(\partial_tF)_{;ij}-(\partial_t\Gamma_{ij}^k)\partial_kF\\
%&=\xi_{;ij}+\left\{\frac{ p }{2}\left(\partial_i\mathcal{H}\delta_j^k+\partial_j\mathcal{H}\delta_i^k-g^{kl}\partial_lg_{ij}\right)-\frac{1}{2}( A ^k_{j;i}+ A ^k_{i;j}+g^{kl} A _{ij;l})\right\}\partial_kF.
%\end{align*}
%We may also calculate $\partial_tF_{;ij}$ in a different way:
%\begin{align*}
%\partial_tF_{;ij}=&\partial_t(g_{ij}\xi+ C _{ij}^k\partial_kF)\\
%=&\left(- p  \mathcal{H}g_{ij}- A _{ij}\right)\xi+g_{ij}(- p  g^{ij}\partial_i\mathcal{H}\partial_jF+ p  \mathcal{H}\xi)\\
%&+(\partial_t  C _{ij}^k)\partial_kF- C _{ij}^l A _l^k\partial_kF.
%\end{align*}
%Putting these together gives
%\begin{align*}
%\partial_t C _{ij}^k=&-\frac{1}{2} A ^l_{i} C _{lj}^k-\frac{1}{2} A _j^l C _{li}^k-\frac{1}{2}g^{kl} A _{ij;l}+ C _{ij}^l A _l^k+ p  g_{ij}g^{kl}\partial_l\mathcal{H}\\
%&+\frac{ p }{2}(\partial_i\mathcal{H}\delta_j^k+\partial_j\mathcal{H}\delta_i^k-g^{kl}\partial_lg_{ij}).
%\end{align*}
%To obtain the evolution equation of $ C _{ijk}$ note that
%$\partial_t C _{ijl}=\partial_t(g_{kl} C _{ij}^k).$
\end{proof}
%\begin{lemma} The following identities hold
%\begin{align*}
%\Delta  C _{ilm}=&\frac{1}{2}g_{im}\partial_l\mathcal{H}+\frac{1}{2}g_{lm}\partial_i\mathcal{H}+\frac{1}{2}g_{il}\partial_m\mathcal{H}\\
%&+\frac{n+2}{2}( A _m^k C _{kil}- A _{il;m})+\frac{1}{2}\mathcal{H} C _{ilm}\\
%&-2 C _{ik}^r C _{mr}^p C _{pl}^k+ C _{im}^r C _{kr}^p C _{pl}^k+ C _{lm}^r C _{kr}^p C _{pi}^k+ C _{jm}^r C _{rp}^j C _{il}^p\\
%&+\frac{ p (n+2)-1}{2}\left\{(\partial_i \ln K)_{;lm}-\frac{1}{2}g_{lm} A _i^k\partial_k\ln K-\frac{1}{2}g_{im} A _l^k\partial_k\ln K\right\}\\
%&-\frac{ p (n+2)-1}{2}\left\{ C _{il}^p C _{mp}^k\partial_k\ln K+\frac{1}{2} A _{mi}\partial_l\ln K+\frac{1}{2} A _{ml}\partial_i\ln K\right\},
%\end{align*}
%and
%\begin{align*}
%\Delta  A _{ij}=&
%\end{align*}
%\end{lemma}
%\begin{proof}


%\end{proof}
\begin{theorem} We have
%\begin{align*}
%\partial_t| C |^2=&
%\end{align*}
%and
\begin{align*}
\partial_t\mathcal{H}= |p| \Delta \mathcal{H}+\frac{p}{|p|}| A |^2+ |p|  \mathcal{H}^2+ |p|  g^{lk}\partial_l\mathcal{H} \operatorname{tr}_{12}( C _{ijk}).
\end{align*}
\end{theorem}
\begin{proof}
We will first calculate the evolution equation of $ A _{ij}.$ Recall that $$\partial_i\xi=- A _i^k\partial_kF.$$ Thus
\begin{align*}
\frac{p}{|p|}\left(- p  g^{ij}\partial_i\mathcal{H}\partial_jF+ p  \mathcal{H}\xi\right)_i&=\partial_t\xi_i=-(\partial_t A _i^k)\partial_kF+\frac{p}{|p|} A _i^k A _j^k\partial_jF.
\end{align*}
We calculate
\begin{align*}
\partial_t A _i^k=\frac{p}{|p|}( A _i^m A _m^k+ p  \mathcal{H}_{;mi}g^{mk}+ p
g^{nm}\partial_n \mathcal{H} C _{mi}^k+ p  \mathcal{H} A _i^k).
\end{align*}
Therefore,
\begin{align*}
\partial_t A _{ij}&=\partial_t( A _i^kg_{kj})\\
&=\frac{p}{|p|}\left(\left( A _i^m A _m^k+ p  \mathcal{H}_{;mi}g^{mk}+ p
g^{nm}\partial_n \mathcal{H} C _{mi}^k+ p  \mathcal{H} A _i^k\right)g_{kj}- A _i^k
( A _{kj}+ p  \mathcal{H}g_{kj})\right).
\end{align*}
Rearranging terms gives
\[\partial_t  A _{ij}= |p|  (\mathcal{H}_{;ij}+  \partial_l\mathcal{H}  C _{ij}^l)\]
Therefore,
\begin{align*}
\partial_t\mathcal{H}&=|p|g^{ij}( \mathcal{H}_{;ij}+  \partial_l \mathcal{H} C ^l_{ij})+\frac{p}{|p|}( A ^{ij}+ p  \mathcal{H}g^{ij}) A _{ij}\\
&= |p| \Delta \mathcal{H}+\frac{p}{|p|}| A |^2+ |p|  \mathcal{H}^2+ |p|  g^{lk}\partial_l\mathcal{H} \operatorname{tr}_{12}( C _{ijk}).
\end{align*}
\end{proof}
\begin{theorem}
\[\left\{
  \begin{array}{ll}
    \partial_t \left(K^{ p }t^{\frac{n p }{n p +1}}\right)> 0, & p>0; \\
    \partial_t \left(K^{ p }t^{\frac{n |p| }{n |p| -1}}\right)<0, & -\frac{1}{n}<p<0.
  \end{array}
\right.
\]
\end{theorem}
\begin{proof}
Suppose $p>0.$
By Lemma 1 we have
\begin{align*}
\partial_t \left(K^{ p }t^{\frac{n p }{n p +1}}\right)&= p  t^{\frac{n p }{n p +1}-1}K^{ p }\left(t\mathcal{H}+\frac{n}{n p +1}\right).
\end{align*}
Thus it suffices to show that $Q:=t\mathcal{H}+\frac{n}{n p +1}$ is always positive:
\begin{align*}
\partial_t Q&= p\Delta Q+ p  g^{lk}\partial_lQ \operatorname{tr}_{12}( C _{ijk})+\mathcal{H}+t| A |^2+  tp \mathcal{H}^2\\
&\geq  |p| \Delta Q+ |p|  g^{lk}\partial_lQ \operatorname{tr}_{12}( C _{ijk})+\mathcal{H}+t\frac{n p +1}{n} \mathcal{H}^2\\
&= p \Delta Q+ p  g^{lk}\partial_lQ \operatorname{tr}_{12}( C _{ijk})+\frac{n p +1}{n}\mathcal{H}Q.
\end{align*}
Therefore, by the maximum principle, $Q$ is always positive.


Suppose $-\frac{1}{n}<p<0.$ We have
\begin{align*}
\partial_t \left(K^{ p }t^{\frac{n |p| }{n |p| -1}}\right)&= |p|  t^{\frac{n |p| }{n |p| -1}-1}K^{ p }\left(t\mathcal{H}+\frac{n}{n |p| -1}\right).
\end{align*}
Thus it suffices to show that $Q:=t\mathcal{H}+\frac{n}{n |p| -1}$ is always negative:
\begin{align*}
\partial_t Q&= |p| \Delta Q+ |p|  g^{lk}\partial_lQ \operatorname{tr}_{12}( C _{ijk})+\mathcal{H}-t| A |^2+ |p|  t \mathcal{H}^2\\
&\leq  |p| \Delta Q+ |p|  g^{lk}\partial_lQ \operatorname{tr}_{12}( C _{ijk})+\mathcal{H}+\frac{n |p| -1}{n}t \mathcal{H}^2\\
&= |p| \Delta Q+ |p|  g^{lk}\partial_lQ \operatorname{tr}_{12}( C _{ijk})+\frac{n |p| -1}{n}\mathcal{H}Q.
\end{align*}
Therefore, by the maximum principle, $Q$ is negative positive.
\end{proof}
\begin{corollary}
The only compact strictly convex ancient solutions of the $p$-GCF for $-\frac{1}{n}<p<0$ are shrinking spheres.
\end{corollary}
\begin{proof}
In the previous theorem we have shown
that $t\mathcal{H}+\frac{n}{n |p| -1}<0$. Therefore, for any $N>0$ we have
\[\mathcal{H}(\cdot,t)+\frac{n}{(n |p| -1)(t-N)}<0\]
for all $t\in [N,\infty).$
Allowing $N\to -\infty$ implies that compact strictly convex ancient solutions satisfy $\mathcal{H}\leq 0.$ A direct calculation shows that
\[\mathcal{H}=h^{ij}(\nabla_{ij}^2K^p+g_{ij}K^{p}).\]
Here $g$ and $\nabla$ are the standard metric and connection of the unit sphere. The normalized solutions must converge forward in time to a self-similar solutions satisfying $\langle F,\nu \rangle=K^{p}$; therefore, $\mathcal{H}(\cdot,t)\geq 0$ if $t$ large enough. Since the hypersurfaces are strictly convex, we conclude that  $\mathcal{H}\equiv 0$ if $t$ large enough and so the hypersurfaces are spheres.
\end{proof}
To get the usual Harnack estimate, we note if we define a time-dependent diffeomorphism $\varphi: M^n\to M^n$ by
\begin{align}
\partial_t\varphi^k=h^{kl}\partial_l K^{ p },
\end{align}
then $\bar{F}(x,t):=F(\varphi(x,t),t):M^{n}\to R^{n+1}$ satisfies
\begin{align}
\partial_t\bar{F}=\bar{K}^{ p }\nu,
\end{align}
where $\bar{K}(x,t):=K(\varphi(x,t),t)$. So we have
\begin{align}
\partial_t\bar{K}^{ p }-h^{kl}\partial_k\bar{K}^{ p }\partial_l\bar{K}^{ p }+\frac{n p }{(n p +1)t}\geq 0.
\end{align}

\bibliographystyle{amsplain}
\bibliography{Bibliography.bib}

\end{document}

 We mention that the cubic form is totally symmetric in all three indices and is trace free with respect to each of  two indices.

Define \[P_{ij}:=\tilde{\nabla}_i\tilde{\nabla}_jS^{-1}+g_{ij}S^{-1},\]
where $g_{ij}$ and $\tilde{\nabla}$ are the metric and the connection of the unit sphere and $S^{-1}$ is the Gauss curvature considered as a function on the unit sphere. First, we would like to calculate the evolution equation of $P_{ij}.$
To do so, we recall that
\[\mathfrak{r}_{ij}=\tilde{\nabla}_i\tilde{\nabla}_jh+g_{ij}h\]
\[\partial_th=-S^{-1},\]
\[S=\det(\mathfrak{r}_{ij})/\det(g_{ij}),\]
\[\partial_t \mathfrak{r}_{ij}=-P_{ij},\]
\[\partial_t S^{-1}=S^{-1}\mathfrak{r}^{kl}P_{kl}.\]
Thus we may calculate
\begin{align*}
\partial_tP_{aj}=&\tilde{\nabla}_a\tilde{\nabla}_j(S^{-1}\mathfrak{r}^{kl}P_{kl})+
g_{aj}S^{-1}\mathfrak{r}^{kl}P_{kl}\\
=&\tilde{\nabla}_a(\mathfrak{r}^{kl}P_{kl}\tilde{\nabla}_jS^{-1}+S^{-1}P_{kl}\tilde{\nabla}_j\mathfrak{r}^{kl}+S^{-1}\mathfrak{r}^{kl}\tilde{\nabla}_jP_{kl})+g_{aj}S^{-1}\mathfrak{r}^{kl}P_{kl}\\
=&\mathfrak{r}^{kl}P_{kl}P_{aj}+\mathfrak{r}^{kl}\tilde{\nabla}_aP_{kl}\tilde{\nabla}_jS^{-1}-\mathfrak{r}^{kr}\mathfrak{r}^{ls}P_{kl}\tilde{\nabla}_jS^{-1}\tilde{\nabla}_a\mathfrak{r}_{rs}\\
&+S^{-1}P_{kl}\tilde{\nabla}_a\tilde{\nabla}_j\mathfrak{r}^{kl}-\mathfrak{r}^{kr}\mathfrak{r}^{ls}P_{kl}\tilde{\nabla}_j\mathfrak{r}_{rs}\tilde{\nabla}_aS^{-1}-S^{-1}\mathfrak{r}^{kr}\mathfrak{r}^{ls}\tilde{\nabla}_aP_{kl}\tilde{\nabla}_j\mathfrak{r}_{rs}\\
&+S^{-1}\mathfrak{r}^{kl}\tilde{\nabla}_a\tilde{\nabla}_jP_{kl}+\mathfrak{r}^{kl}\tilde{\nabla}_jP_{kl}\tilde{\nabla}_aS^{-1}-S^{-1}\mathfrak{r}^{kr}\mathfrak{r}^{ls}\tilde{\nabla}_jP_{kl}\tilde{\nabla}_a\mathfrak{r}_{rs}.
\end{align*}

\begin{align*}
\partial_t \mathfrak{r}^{ia}=\mathfrak{r}^{ic}\mathfrak{r}^{ab}P_{bc}.
\end{align*}
Next we would like to calculate the evolution equation
of
$Q_j^i:=\mathfrak{r}^{ia}P_{aj}.$
\begin{align*}
\partial_t Q_j^i=(Q^2)_i^j+\mathfrak{r}^{ia}\partial_tP_{aj}
\end{align*}
\begin{align*}
\\mathfrak{r}artial_tP_{aj}=&S^{-2}S^{kl}P_{kl}g_{aj}-6S^{-4}S^{kl}P_{kl}\tilde{\nabla}_aS\tilde{\nabla}_jS-2S^{-3}S^{kl}P_{kl}\tilde{\nabla}_a\tilde{\nabla}_jS\\
&-2S^{-3}S^{kl,mn}P_{kl}\tilde{\nabla}_a\mathfrak{r}_{mn}\tilde{\nabla}_jS-2S^{-3}S^{kl}\tilde{\nabla}_aP_{kl}\tilde{\nabla}_jS\\
&-2S^{-3}S^{kl,mn}\tilde{\nabla}_j\mathfrak{r}_{mn}P_{kl}\tilde{\nabla}_aS+S^{-2}S^{kl,mn,rs}P_{kl}\tilde{\nabla}_j\mathfrak{r}_{mn}\tilde{\nabla}_a\mathfrak{r}_{rs}\\
&+S^{-2}S^{kl,mn}\tilde{\nabla}_aP_{kl}\tilde{\nabla}_j\mathfrak{r}_{mn}+S^{-2}S^{kl,mn}P_{kl}\tilde{\nabla}_a\tilde{\nabla}_j\mathfrak{r}_{mn}\\
&-2S^{-3}S^{kl}\tilde{\nabla}_jP_{kl}\tilde{\nabla}_aS+S^{-2}S^{kl,mn}\tilde{\nabla}_jP_{kl}\tilde{\nabla}_a\mathfrak{r}_{mn}+S^{-2}S^{kl}\tilde{\nabla}_a\tilde{\nabla}_jP_{kl}.
\end{align*} 