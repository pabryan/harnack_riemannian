\documentclass{amsart}
\usepackage[ocgcolorlinks,linktoc=all]{hyperref}
\usepackage{cancel}
\hypersetup{citecolor=blue,linkcolor=red}
\newtheorem{theorem}{Theorem}
\newtheorem*{thmA}{Theorem}
\newtheorem*{thmB}{Theorem}
\newtheorem*{rem}{Remark}
\newtheorem*{thmmain}{Theorem}
\newtheorem{lemma}[theorem]{Lemma}
\newtheorem{proposition}[theorem]{Proposition}
\newtheorem*{propmain}{Proposition}
\newtheorem{corollary}[theorem]{Corollary}
\theoremstyle{definition}
\newtheorem{definition}[theorem]{Definition}
\newtheorem{example}[theorem]{Example}
\newtheorem{xca}[theorem]{Exercise}

\theoremstyle{remark}
\newtheorem{remark}[theorem]{Remark}

\newcommand{\abs}[1]{\lvert#1\rvert}
\newcommand{\ip}[2]{\ensuremath{\langle{#1},{#2}\rangle}}

\DeclareMathOperator{\grad}{grad}
\DeclareMathOperator{\Rm}{Rm}
\DeclareMathOperator{\Tr}{Tr}
\DeclareMathOperator{\hess}{Hess}
\DeclareMathOperator{\ric}{Ric}
\DeclareMathOperator{\diff}{K}
\DeclareMathOperator{\D}{D}
\DeclareMathOperator{\RR}{\mathbb{R}}


\numberwithin{equation}{section}

\newcommand{\blankbox}[2]{%
  \parbox{\columnwidth}{\centering
    \setlength{\fboxsep}{0pt}%
    \fbox{\raisebox{0pt}[#2]{\hspace{#1}}}%
  }%
}

\begin{document}

\title[]
 {}

\curraddr{}
\email{}
\date{\today}

\dedicatory{}
\subjclass[2010]{}
\keywords{}

\begin{abstract}

\end{abstract}

\maketitle
Let $M^n$ be an $n$-dimensional smooth submanifold of $(N^{n+1},\ip{}{})$ given by an embedding $F_0.$ A one-parameter family of immersions $F\colon M^n\times [0,T)\to (N^{n+1},\ip{}{})$ is said to be a solution of the $f$-curvature flow if
\begin{align}
\partial_tF = \xi,\quad F(\cdot, 0) = F_0(\cdot).
\end{align}
Here $\xi $ is a smooth transverse inward-pointing vector field defined by
\begin{align}
\xi := -F_{\ast}\grad_h f + f\nu,
\end{align}
where $f$ is a smooth, positive, strictly increasing and symmetric function of the principle curvatures, $\nu$ is a smooth inner unit normal vector field, and $\grad_h f$ is the gradient of $f$ with respect to the second fundamental $h$ induced by $\nu,$ namely
\[
h(X, \grad_h f) = X(f)
\]
for all vectors $X$ tangent to $M$. In what follows, for convenience, we set $W := -\grad_h f$ so that $\xi = F_{\ast} W + f \nu$.


\section{structure equations}

We make use of the theory of affine immersions as described in \cite{MR1311248}.

Let $\bar{g}$ be the induced metric by the immersion. The Gauss formula is given by
\begin{align}\label{gauss equ}
\D_YF_{\ast}(X)=F_{\ast}(\bar{\nabla}_YX)+h(X,Y)\nu.
\end{align}
Define the non-degenerate metric, referred to here as the \emph{affine metric},
\begin{align}
g(X,Y):=\frac{h(X,Y)}{ f }.
\end{align}
Then, for any smooth function $\varphi : M \to \RR$, $X(\varphi) = h(X, \grad_h \varphi) = g(X, f \grad_h \varphi)$ implies that, $\grad \varphi = f \grad_h \varphi$ where $\grad$ denotes the gradient with respect to $g$. In particular, $\grad_h f = \grad \ln f$ and we may also write $W = -\grad \ln f$.

Note that
\begin{align*}
\D_X \xi &= \D_X (F_{\ast}(W) + f \nu) \\
&= \D_X F_{\ast}(W) + X(f)\nu + f \D_X \nu \\
&= F_{\ast}(\bar{\nabla}_X W) - h(X, \grad_h f) \nu + X(f) \nu + f \D_X \nu \\
&= F_{\ast} (\bar{\nabla}_X W) + f \D_X \nu,
\end{align*}
which is tangential since $\nu$ has unit length. Thus, we may define an endomorphism of $TM$, the \emph{affine shape operator} $A$ with respect to $\xi$, by
\begin{align}
\D_X\xi := -F_{\ast}(A(X))
\end{align}
for $X\in TM$. Equivalently,
\[
A(X) =- \bar{\nabla}_X W - f F_{\ast}^{-1} \D_X \nu = -\bar{\nabla}_X W + f \mathcal{W} (X)
\]
where $\mathcal{W}$ is the usual Weingarten map (shape operator). We also have the \emph{affine second fundamental form}, a tensor of type $(0, 2)$, by
\[
B(X, Y) = g(A(X), Y).
\]
In general, the presence of ambient curvature $\widehat{\Rm}$ implies that $A$ is not self-adjoint with respect to $g$ (see \eqref{eq:structure2} below), and hence $B$ is not symmetric. For this reason, we typically will not use $B$, preferring $g(A(X), Y)$ emphasising the lack of symmetry. We only mention $B$ here to emphasise the lack of symmetry, compared with the usual second fundamental form, $\bar{g}(-\D_{\nu} X, Y)$ which is symmetric, and the Weingarten map $-\D_{\nu} X$ which is thus self-adjoint with respect to $\bar{g}$.

The vector field $\xi$ induces a torsion free connection $\tilde{\nabla}$:
\begin{align}\label{gauss equ2}
\D_YF_{\ast}(X)=F_{\ast}(\tilde{\nabla}_YX)+g(X,Y)\xi.
\end{align}
This connection is well defined (i.e. $F_{\ast}(\tilde{\nabla}_Y X)$ is tangent to $F(M)$) since for all $X,Y\in TM$ we have
\begin{align}
F_{\ast}(\tilde{\nabla}_Y X) &= F_{\ast}(\bar{\nabla}_Y X) + h(X, Y) \nu - g(X, Y) (F_{\ast}(W) + f \nu) \\
&= F_{\ast}(\bar{\nabla}_YX)-g(X,Y)F_{\ast}(W) \nonumber.
\end{align}
In other words, $\tilde{\nabla}_Y X$ is the uniquely determined tangential component of $D_Y F_{\ast} X$ in the splitting $F^{\ast} TN \simeq TM \oplus \RR \xi$ where $\RR\xi$ denotes the line bundle over $M$ determined by $\xi$ and $F^{\ast} TN$ denotes the pull-back bundle of $TN$ by $F$.

The connection $\tilde{\nabla}$ is not necessarily the Levi-Civita connection for $g$. Let $\nabla$ be the Levi-Civita connection for $g$; that is $\nabla$ is the unique torsion free connection satisfying $\nabla g=0.$ Define the difference tensor
\begin{equation}
\label{eq:difftensor}
\diff_YX := \tilde{\nabla}_{Y}X - \nabla_{Y}X.
\end{equation}
Since both $\tilde{\nabla}$ and $\nabla$ are torsion free, we have $\diff$ is symmetric: $\diff_YX = \diff_XY.$

From (\ref{gauss equ2}) we get,
\begin{align}\label{gauss equ3}
\hess F(X,Y):=\D_YF_{\ast}(X)-F_{\ast}(\nabla_YX) = F_{\ast}(\diff_YX) + g(X,Y)\xi.
\end{align}

Let us recall the \emph{structure equations} of an affine immersion (see the proof of \cite[Section II, Theorem 2.1]{MR1311248} adjusted to include the ambient curvature $\widehat{Rm}$ or \cite[p. 197 equations (N1.6)--(N1.9)]{MR1311248} in the codimension one case):

\begin{align}
\label{eq:structure1}
\widehat{Rm} (X, Y) Z =& Rm(X, Y) Z + g(X, Z) A Y - g(Y, Z)A X \\
&+ \left[(\tilde{\nabla}_X g) (Y, Z) - (\tilde{\nabla}_Y g)( X, Z)\right] \xi \nonumber,\\
\label{eq:structure2}
\widehat{Rm} (X, Y) \xi &= (\tilde{\nabla}_Y A) X - (\tilde{\nabla}_X A) Y + \left[g(AX, Y) - g(X, AY)\right]\xi.
\end{align}

The \emph{cubic tensor} is defined by
\begin{align}
C(X,Y,Z) = (\tilde{\nabla}_X g) (Y,Z).
\end{align}

We have the following useful lemma for $C$ (see also \cite[Section II, Proposition 4.1]{MR1311248}):

\begin{lemma}
\[
C(X, Y, Z) =  -2g(\diff_Z Y, X) + \left[\widehat{Rm}(X, Y)Z + \widehat{Rm}(X, Z)Y\right]^{\xi}.
\]
\end{lemma}

\begin{proof}
First, since $g$ is symmetric, $C(X, Y, Z) = C(X, Z, Y)$:
\[
\begin{split}
C(X, Y, Z) - C(X, Z, Y) &= (\tilde{\nabla}_X g) (Y, Z) - (\tilde{\nabla}_X g) (Z, Y) \\
&= \tilde{\nabla}_X \left[g(Y, Z) - g(Z, Y)\right] \\
&\quad - g(\tilde{\nabla}_X Y, Z) - g(Y, \tilde{\nabla}_X Z) + g(\tilde{\nabla}_X Z, Y) + g(Z, \tilde{\nabla}_X Y) \\
&= 0.
\end{split}
\]

But if $\widehat{Rm} \ne 0$, $C$ may not be symmetric in the first two slots: using the structure equations,
\[
C(X, Y, Z) = C(Y, X, Z) + [\widehat{Rm}(X, Y)Z]^{\xi},
\]
where the superscript $\xi$ denotes the $\xi$ component in the splitting, $F^{\ast} TN \simeq TM \oplus \RR \xi$. Applying the structure equation to $C(Y, X, Z) = C(Y, Z, X)$, we obtain,
\begin{equation}
\label{eq:cubic_permutation1}
C(X, Y, Z) = C(Z, Y, X) + \left[\widehat{Rm}(Y, Z) X + \widehat{Rm}(X, Y) Z\right]^{\xi}.
\end{equation}
On the other hand, since $\diff_X \varphi = (\tilde{\nabla}_X - \nabla_X) \varphi = 0$ for any smooth function $\varphi$, applying $\diff_X$ to $g$ as a derivation gives,
\[
(\diff_X g) (Y, Z) = \diff_X (g(Y, Z)) - g(\diff_X Y, Z) - g(Y, \diff_X Z) = -g(\diff_X Y, Z) - g(Y, \diff_X Z).
\]
Now using $\nabla g \equiv 0$, gives
\begin{equation}
\label{eq:cubic_derivation_identity}
C(X, Y, Z) = (\tilde{\nabla}_X g) (Y, Z) = (\diff_X g) (Y, Z) + (\nabla_X g) (Y, Z) = -g(\diff_X Y, Z) - g(Y, \diff_X Z).
\end{equation}

Thus from \eqref{eq:cubic_permutation1}, \eqref{eq:cubic_derivation_identity} we obtain,
\begin{equation}
\label{eq:cubic_identity}
\begin{split} 
-g(\diff_X Y, Z) - g(Y, \diff_X Z) &= C(Z, Y, X) + \left[\widehat{Rm}(Y, Z) X + \widehat{Rm}(X, Y)Z\right]^\xi \\
&= -g(\diff_Z Y, X) - g(Y, \diff_Z X) + \left[\widehat{Rm}(Y, Z) X + \widehat{Rm}(X, Y)Z\right]^\xi
\end{split}
\end{equation}
Since $\diff_X Z = \diff_Z X$ this yields,
\[
-g(\diff_X Y, Z) = - g(\diff_Z Y, X) + \left[\widehat{Rm}(Y, Z) X + \widehat{Rm}(X, Y)Z\right]^\xi
\]
When applied to $g(\diff_X Z, Y)$ we obtain
\[
-g(\diff_Z X, Y) = - g(\diff_X Z, Y) = - g(\diff_Y Z, X) + \left[\widehat{Rm}(Z, Y) X + \widehat{Rm}(X, Z)Y\right]^\xi
\]
Finally, upon substitution of this identity into equation \eqref{eq:cubic_identity} we complete the proof with,
\begin{equation}
\begin{split}
C(X, Y, Z) &= -g(\diff_Z Y, X) - g(\diff_Z X, Y) + \left[\widehat{Rm}(Y, Z) X + \widehat{Rm}(X, Y)Z\right]^\xi \\
&= -g(\diff_Z Y, X) - g(\diff_Y Z, X) \\
&\quad + \left[\widehat{Rm}(Y, Z) X + \widehat{Rm}(X, Y)Z + \widehat{Rm}(Z, Y) X + \widehat{Rm}(X, Z)Y\right]^\xi \\
&= - 2g(\diff_Y Z, X) + \left[\widehat{Rm}(X, Y)Z + \widehat{Rm}(X, Z)Y\right]^{\xi}
\end{split}
\end{equation}
where we used the first anti-symmetry of $\widehat{Rm}$ in the last line.
\end{proof}

The second covariant derivatives of $ \xi$ is given by
\begin{align}
\hess\xi(X,Y)&=\D_Y\D_X\xi-\D_{\nabla_YX}\xi\\
&=-\D_YF_{\ast}(AX)+F_{\ast}(A\nabla_YX)\nonumber\\
&=-F_{\ast}(\tilde{\nabla}_{Y}AX)-g(A X,Y)\xi+F_{\ast}(A\nabla_YX)\nonumber\\
&=-F_{\ast}(\nabla_{Y}AX)-F_{\ast}(\diff_{Y}AX)-g(A X,Y)\xi+F_{\ast}(A\nabla_YX)\nonumber\\
&=-F_{\ast}((\nabla_Y A)X)-F_{\ast}(\diff_{Y}AX)-g(A X,Y)\xi.\nonumber
\end{align}

On the other hand, we see that if $\widehat{\Rm}$ does not vanish, then $\hess \xi$ is not symmetric:
\begin{align}
\widehat{Rm}(X,Y)\xi&=\D^2_{X,Y}\xi-\D^2_{Y,X}\xi\\
&=\hess\xi(X,Y)-\hess\xi(Y,X)-\D_{\D_YX-\D_XY}\xi-\D_{\nabla_XY-\nabla_YX}\xi\nonumber\\
&=\hess\xi(X,Y)-\hess\xi(Y,X).\nonumber
\end{align}

\section{evolution equations}

Perhaps the most important aspect of our approach is the following lemma, exhibiting crucial properties of our chosen parametrisation.

\begin{lemma}
\begin{align}
\D_t \nu & = 0, \\
\D_t \bar{g} &= 0.
\end{align}
\end{lemma}

\begin{proof}
As $\ip{\nu}{\nu} = 1$, $\D_{\xi} \nu$ is tangential, and for any tangent vector $X$,
\[
\begin{split}
\ip{\D_{\xi} \nu}{F_{\ast} X} &= D_{\xi} \ip{\nu}{F_{\ast}X} - \ip{\nu}{D_{\xi} F_{\ast} X} \\
&= -\ip{\nu}{\D_{X}\xi} \\
&= \ip{\nu}{A(X)} = 0.
\end{split}
\]

For the metric,
\[
\begin{split}
\left(\D_t \bar{g}\right) (X, Y) &= \D_t (\bar{g}(X, Y)) - \bar{g}(\D_t X, Y) - \bar{g}(X, \D_t Y) \\
&= \D_t (\bar{g}(X, Y)) + \bar{g}(A(X), Y) + \bar{g}(X, A(Y)).
\end{split}
\]
On the other hand,
\[
\begin{split}
\D_t (\bar{g}(X, Y)) &= \D_t \ip{F_{\ast} X}{F_{\ast} Y} \\
&= -\ip{F_{\ast} A(X)}{F_{\ast} Y} - \ip{F_{\ast} X}{F_{\ast} A(Y)}
\end{split}
\]
and the result follows since $\bar{g} = F^{\ast} \ip{}{}$.
\end{proof}

The next lemma contains other basic evolution equations. It will be convenient at this stage to express these equations in terms of $u = \partial_t \ln f$.
\begin{lemma}
\begin{align*}
(\D_t h) (X, Y) &= f g(X, A(Y)) + \ip{\widehat{\Rm}(\xi, X) Y}{\nu} =  f g(A(X), Y) + \ip{\widehat{\Rm}(\xi, Y) X}{\nu}, \\
(\D_t g) (X, Y) &= g(X, A(Y)) - u g(X, Y) + \frac{1}{f} \ip{\widehat{\Rm}(\xi, X) Y}{\nu}, \\
\D_t \grad \ln f &= \grad u + u \grad \ln f + R(\grad \ln f) \\
\intertext{where $g(R({\grad \ln f}), X) = \frac{1}{f} \ip{\widehat{\Rm}(\xi, \grad \ln f) \nu}{X}$ for any vector field $X$,}
\D_t \xi &= u\xi - \grad u - R(\grad \ln f), \\
(\D_t A)(X) &= A^2(X) + u A(X) + \nabla_X (\grad u) + \diff_X (\grad u) \\
&\quad + \widehat{\Rm} (\xi, X) \xi + \D_X (R(\grad \ln f)), \\
(\D_t B) (X, Y) &= g(A(X), A(Y)) + g(A^2(X), Y) + \hess u (X, Y) -\frac{1}{2} C(Y, \grad u, X) \\
&\quad +\frac{1}{f} \ip{\hat{\Rm} (\xi, A(X))Y}{\nu} + g(\widehat{\Rm}(\xi, X)\xi  + \D_X (R(\grad_h f)), Y) \\
&\quad -\frac{1}{2} \left[\widehat{\Rm} (Y, \grad u)X +  \widehat{\Rm} (Y, X) \grad u\right]^{\xi}.
\end{align*}
\end{lemma}

\begin{proof}
Using $h(X,Y) = \ip{\D_Y X}{\nu} = -\ip{X}{D_Y \nu}$ and $\D_t\nu = 0$, we calculate
\[
\begin{split}
(\D_t h) (X, Y) &= \D_t (h(X, Y)) - h(\D_t X, Y) - h(X, D_t Y) \\
&= \ip{\D_t \D_Y X}{\nu} + f g(A(X), Y) + f g(X, A(Y)) \\
&= \ip{\D_Y \D_{\xi} X}{\nu} + \ip{\widehat{\Rm}(\xi, X) Y}{\nu} + f g(A(X), Y) + f g(X, A(Y))\\
&= \D_Y \ip{\D_{\xi} X}{\nu} - \ip{\D_{\xi} X}{D_Y \nu} + f g(A(X), Y) + f g(X, A(Y)) + \ip{\widehat{\Rm}(\xi, X) Y}{\nu} \\
&= - h(A(X), Y) + f g(A(X), Y) + f g(X, A(Y)) + \ip{\widehat{\Rm}(\xi, X) Y}{\nu} \\
&= fg(X, A(Y)) + \ip{\widehat{\Rm}(\xi, X) Y}{\nu},
\end{split}
\]
giving the first expression for $\D_t h$. The second follows from the symmetry of $\D_t h$.

The metric now follows easily,
\[
\begin{split}
(\D_t g) (X, Y) &= -\frac{1}{f^2} (\D_t f) h(X, Y) + \frac{1}{f} (\D_t h) (X, Y) \\
&= -u g(X, Y) + g(X, A(Y)) + \frac{1}{f} \ip{\widehat{\Rm}(\xi, X) Y}{\nu}.
\end{split}
\]

To compute $\D_t \grad \ln f$, we use the defining equation, $g(\grad \ln f, X) = X(\ln f) = \D_X \ln f$ to compute,
\[
\begin{split}
g(\grad u, X) &= g(\grad \D_t \ln f, X) = \D_X \D_t \ln f = \D_t \D_X \ln f \\
&=  \D_t (g(\grad \ln f, X)) \\
&= (\D_t g) (\grad \ln f, X)) + g(\D_t \grad \ln f, X) + g(\grad \ln f, \D_t X) \\
&= -ug(\grad\ln f, X) + g(\grad \ln f, A(X)) + \frac{1}{f}\ip{\widehat{\Rm}(\xi, \grad \ln f)X}{\nu} \\
&\quad  + g(\D_t \grad \ln f, X) - g(\grad \ln f, A(X)) \\
&= - u g(\grad \ln f, X) + g(\D_t \grad \ln f, X) - \frac{1}{f} \ip{\widehat{\Rm}(\xi, \grad_h f)\nu}{X}.
\end{split}
\]
Rearranging gives,
\[
g(\D_t \grad \ln f - u \grad\ln f - \grad u, X) = \frac{1}{f} \ip{\widehat{\Rm}(\xi, \grad \ln f) \nu}{X}.
\]
Since this is true for any $X$, the desired result follows.

The evolution of $\xi$ follows readily,
\[
\begin{split}
\D_t \xi &= \D_t (-\grad \ln f + f \nu) \\
& = -\grad u - u \grad \ln f + (f \D_t \ln f) \nu - R(\grad \ln f) \\
&= u\xi - \grad u - R(\grad \ln f).
\end{split}
\]

This allows us to compute,
\[
\begin{split}
(\D_t A)(X) &= \D_t (A(X)) - A(\D_t X) \\
&= - \D_t \D_X \xi + A^2(X) \\
&= -\D_X \D_t \xi + A^2(X) + \widehat{\Rm} (\xi, X) \xi \\
&= -\D_X \left(u\xi - \grad u - R(\grad \ln f)\right) + A^2(X) + \widehat{\Rm} (\xi, X) \xi \\
&= A^2(X) + u A(X) -(\D_X u)\xi + \D_X (\grad u) + \widehat{\Rm} (\xi, X) \xi + \D_X (R(\grad \ln f)).
\end{split}
\]
Then, we observe that by equation \eqref{gauss equ2} defining $\tilde{\nabla}$ we have,
\[
\begin{split}
-(\D_X u)\xi + \D_X (\grad u) &= -(\D_X u)\xi + \tilde{\nabla}_X (\grad u) + g(\grad u, X) \xi \\
&= \tilde{\nabla}_X (\grad u) \\
&= \nabla_X (\grad u) + \diff_X (\grad u).
\end{split}
\]

Finally, we have
\[
\begin{split}
(\D_t B) (X, Y) &= \D_t (B(X, Y)) - B(\D_t X, Y) - B(X, \D_t Y) \\
&= \D_t (g(A(X), Y)) - g(A(\D_tX), Y) - g(A(X), D_t Y) \\
&= (\D_t g)(A(X), Y) + g(\D_t(A(X)), Y) + g(A(X), \D_t Y) \\
&\quad - g(A(\D_tX), Y) - g(A(X), D_t Y) \\
&= (\D_t g)(A(X), Y) + g((\D_t A)(X), Y) \\
&= g(A(X), A(Y)) - u g(A(X), Y) + \frac{1}{f} \ip{\hat{\Rm} (\xi, A(X))Y}{\nu} \\
&\quad + g(A^2(X) + u A(X) + \nabla_X \grad u + \diff_X \grad u, Y) \\
&\quad + g(\widehat{\Rm}(\xi, X)\xi, Y) + g(\D_X (R(\grad \ln f)), Y) \\
&= g(A(X), A(Y)) + g(A^2(X), Y) + \hess u (X, Y) -\frac{1}{2} C(Y, \grad u, X) \\
&\quad +\frac{1}{f} \ip{\hat{\Rm} (\xi, A(X))Y}{\nu} + g(\widehat{\Rm}(\xi, X)\xi  + \D_X (R(\grad_h f)), Y) \\
&\quad -\frac{1}{2} \left[\widehat{\Rm} (Y, \grad u)X +  \widehat{\Rm} (Y, X) \grad u\right]^{\xi}.
\end{split}
\]
\end{proof}

\section{The Harnack Inequality}

\begin{theorem}
\[\left\{
  \begin{array}{ll}
    \partial_t \left(\varphi(\nu)K^pt^{\frac{n p }{n p +1}}\right)> 0, & p>0; \\
    \partial_t \left(\varphi(\nu)K^pt^{\frac{n |p| }{n |p| -1}}\right)<0, & -\frac{1}{n}<p<0.
  \end{array}
\right.
\]
\end{theorem}
\begin{proof}
Suppose $p>0.$
By Lemma 1 we have
\begin{align*}
\partial_t \left(ft^{\frac{n p }{n p +1}}\right)&= p  t^{\frac{n p }{n p +1}-1}f\left(t\mathcal{H}+\frac{n}{n p +1}\right).
\end{align*}
Thus it suffices to show that $Q:=t\mathcal{H}+\frac{n}{n p +1}$ is always positive:
\begin{align*}
\partial_t Q&= p\Delta Q+ p  g^{lk}\partial_lQ \Tr_{12}( C _{ijk})+\mathcal{H}+t| A |^2+  tp \mathcal{H}^2\\
&\geq  p \Delta Q+ p  g^{lk}\partial_lQ \Tr_{12}( C _{ijk})+\mathcal{H}+t\frac{n p +1}{n} \mathcal{H}^2\\
&= p \Delta Q+ p  g^{lk}\partial_lQ \Tr_{12}( C _{ijk})+\frac{n p +1}{n}\mathcal{H}Q.
\end{align*}
Therefore, by the maximum principle, $Q$ is always positive.


Suppose $-\frac{1}{n}<p<0.$ We have
\begin{align*}
\partial_t \left(ft^{\frac{n |p| }{n |p| -1}}\right)&= |p|  t^{\frac{n |p| }{n |p| -1}-1}f\left(t\mathcal{H}+\frac{n}{n |p| -1}\right).
\end{align*}
Thus it suffices to show that $Q:=t\mathcal{H}+\frac{n}{n |p| -1}$ is always negative:
\begin{align*}
\partial_t Q&= |p| \Delta Q+ |p|  g^{lk}\partial_lQ \Tr_{12}( C _{ijk})+\mathcal{H}-t| A |^2+ |p|  t \mathcal{H}^2\\
&\leq  |p| \Delta Q+ |p|  g^{lk}\partial_lQ \Tr_{12}( C _{ijk})+\mathcal{H}+\frac{n |p| -1}{n}t \mathcal{H}^2\\
&= |p| \Delta Q+ |p|  g^{lk}\partial_lQ \Tr_{12}( C _{ijk})+\frac{n |p| -1}{n}\mathcal{H}Q.
\end{align*}
Therefore, by the maximum principle, $Q$ is negative positive.
\end{proof}

To get the usual Harnack estimate, we note if we define a time-dependent diffeomorphism $\varphi: M^n\to M^n$ by
\begin{align}
\partial_t\varphi^k=h^{kl}\partial_l K^{ p },
\end{align}
then $\bar{F}(x,t):=F(\varphi(x,t),t):M^{n}\to R^{n+1}$ satisfies
\begin{align}
\partial_t\bar{F}=\bar{K}^{ p }\nu,
\end{align}
where $\bar{K}(x,t):=K(\varphi(x,t),t)$. So we have
\begin{align}
\partial_t\bar{K}^{ p }-h^{kl}\partial_k\bar{K}^{ p }\partial_l\bar{K}^{ p }+\frac{n p }{(n p +1)t}\geq 0.
\end{align}

\bibliographystyle{amsplain}
\bibliography{Bibliography.bib}

\end{document}
