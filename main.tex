\documentclass{amsart}
\usepackage[ocgcolorlinks,linktoc=all]{hyperref}
\usepackage{cancel}
\hypersetup{citecolor=blue,linkcolor=red}
\newtheorem{theorem}{Theorem}
\newtheorem*{thmA}{Theorem}
\newtheorem*{thmB}{Theorem}
\newtheorem*{rem}{Remark}
\newtheorem*{thmmain}{Theorem}
\newtheorem{lemma}[theorem]{Lemma}
\newtheorem{proposition}[theorem]{Proposition}
\newtheorem*{propmain}{Proposition}
\newtheorem{corollary}[theorem]{Corollary}
\theoremstyle{definition}
\newtheorem{definition}[theorem]{Definition}
\newtheorem{example}[theorem]{Example}
\newtheorem{xca}[theorem]{Exercise}

\theoremstyle{remark}
\newtheorem{remark}[theorem]{Remark}

\newcommand{\abs}[1]{\lvert#1\rvert}
\newcommand{\ip}[2]{\ensuremath{\langle{#1},{#2}\rangle}}

\DeclareMathOperator{\grad}{grad}
\DeclareMathOperator{\Rm}{Rm}
\DeclareMathOperator{\Tr}{Tr}
\DeclareMathOperator{\hess}{Hess}
\DeclareMathOperator{\ric}{Ric}
\DeclareMathOperator{\diff}{K}
\DeclareMathOperator{\D}{D}
\DeclareMathOperator{\RR}{\mathbb{R}}


\numberwithin{equation}{section}

\newcommand{\blankbox}[2]{%
  \parbox{\columnwidth}{\centering
    \setlength{\fboxsep}{0pt}%
    \fbox{\raisebox{0pt}[#2]{\hspace{#1}}}%
  }%
}

\begin{document}

\title[]
 {}

\curraddr{}
\email{}
\date{\today}

\dedicatory{}
\subjclass[2010]{}
\keywords{}

\begin{abstract}

\end{abstract}

\maketitle
Let $M^n$ be an $n$-dimensional smooth submanifold of $(N^{n+1},\ip{}{})$ given by an embedding $F_0.$ A one-parameter family of immersions $F\colon M^n\times [0,T)\to (N^{n+1},\ip{}{})$ is said to be a solution of the $f$-curvature flow if
\begin{align}
\partial_tF = \xi,\quad F(\cdot, 0) = F_0(\cdot).
\end{align}
Here $\xi $ is a smooth transverse inward-pointing vector field defined by
\begin{align}
\xi := -F_{\ast}\grad_h f + f\nu,
\end{align}
where $f$ is a smooth, positive, strictly increasing and symmetric function of the principle curvatures, $\nu$ is a smooth inner unit normal vector field, and $\grad_h f$ is the gradient of $f$ with respect to the second fundamental $h$ induced by $\nu,$ namely
\[
h(X, \grad_h f) = X(f)
\]
for all vectors $X$ tangent to $M$. In what follows, for convenience, we set $W := -\grad_h f$ so that $\xi = F_{\ast} W + f \nu$.


\section{structure equations}

We make use of the theory of affine immersions as described in \cite{MR1311248}.

Let $\bar{g}$ be the induced metric by the immersion. The Gauss formula is given by
\begin{align}\label{gauss equ}
\D_YF_{\ast}(X)=F_{\ast}(\bar{\nabla}_YX)+h(X,Y)\nu.
\end{align}
Define the non-degenerate metric, referred to here as the \emph{affine metric},
\begin{align}
g(X,Y):=\frac{h(X,Y)}{ f }.
\end{align}
Note that
\begin{align*}
\D_X \xi &= \D_X (F_{\ast}(W) + f \nu) \\
&= \D_X F_{\ast}(W) + X(f)\nu + f \D_X \nu \\
&= F_{\ast}(\bar{\nabla}_X W) - h(X, \grad_h f) \nu + X(f) \nu + f \D_X \nu \\
&= F_{\ast} (\bar{\nabla}_X W) + f \D_X \nu,
\end{align*}
which is tangential since $\nu$ has unit length. Thus, we may define an endomorphism of $TM$, the \emph{affine shape operator} $A$ with respect to $\xi$, by
\begin{align}
\D_X\xi := -F_{\ast}(A(X))
\end{align}
for $X\in TM$. Equivalently,
\[
A(X) =- \bar{\nabla}_X W - f F_{\ast}^{-1} \D_X \nu.
\]
We also define the \emph{affine second fundamental form}, a tensor of type $(0, 2)$, by
\[
B(X, Y) = g(A(X), Y).
\]
In general, the presence of ambient curvature $\widehat{\Rm}$ implies that $A$ is not self-adjoint with respect to $g$ (see \eqref{eq:structure2} below).

The vector field $\xi$ induces a torsion free connection $\tilde{\nabla}$:
\begin{align}\label{gauss equ2}
\D_YF_{\ast}(X)=F_{\ast}(\tilde{\nabla}_YX)+g(X,Y)\xi.
\end{align}
This connection is well defined (i.e. $F_{\ast}(\tilde{\nabla}_Y X)$ is tangent to $F(M)$) since for all $X,Y\in TM$ we have
\begin{align}
F_{\ast}(\tilde{\nabla}_Y X) &= F_{\ast}(\bar{\nabla}_Y X) + h(X, Y) \nu - g(X, Y) (F_{\ast}(W) + f \nu) \\
&= F_{\ast}(\bar{\nabla}_YX)-g(X,Y)F_{\ast}(W) \nonumber.
\end{align}
In other words, $\tilde{\nabla}_Y X$ is the uniquely determined tangential component of $D_Y F_{\ast} X$ in the splitting $F^{\ast} TN \simeq TM \oplus \RR \xi$ where $\RR\xi$ denotes the line bundle over $M$ determined by $\xi$ and $F^{\ast} TN$ denotes the pull-back bundle of $TN$ by $F$.

The connection $\tilde{\nabla}$ is not necessarily the Levi-Civita connection for $g$. Let $\nabla$ be the Levi-Civita connection for $g$; that is $\nabla$ is the unique torsion free connection satisfying $\nabla g=0.$ Define the difference tensor
\begin{equation}
\label{eq:difftensor}
\diff_YX := \tilde{\nabla}_{Y}X - \nabla_{Y}X.
\end{equation}
Since both $\tilde{\nabla}$ and $\nabla$ are torsion free, we have $\diff$ is symmetric: $\diff_YX = \diff_XY.$

From (\ref{gauss equ2}) we get,
\begin{align}\label{gauss equ3}
\hess F(X,Y):=\D_YF_{\ast}(X)-F_{\ast}(\nabla_YX) = F_{\ast}(\diff_YX) + g(X,Y)\xi.
\end{align}

Let us recall the \emph{structure equations} of an affine immersion (see the proof of \cite[Section II, Theorem 2.1]{MR1311248} adjusted to include the ambient curvature $\widehat{Rm}$ or \cite[p. 197 equations (N1.6)--(N1.9)]{MR1311248} in the codimension one case):

\begin{align}
\label{eq:structure1}
\widehat{Rm} (X, Y) Z =& Rm(X, Y) Z + g(X, Z) A Y - g(Y, Z)A X \\
&+ \left[(\tilde{\nabla}_X g) (Y, Z) - (\tilde{\nabla}_Y g)( X, Z)\right] \xi \nonumber,\\
\label{eq:structure2}
\widehat{Rm} (X, Y) \xi &= (\tilde{\nabla}_Y A) X - (\tilde{\nabla}_X A) Y + \left[g(AX, Y) - g(X, AY)\right]\xi.
\end{align}

The \emph{cubic tensor} is defined by
\begin{align}
C(X,Y,Z) = (\tilde{\nabla}_X g) (Y,Z).
\end{align}

We have the following useful lemma for $C$ (see also \cite[Section II, Proposition 4.1]{MR1311248}):

\begin{lemma}
\[
C(X, Y, Z) =  -2g(\diff_Z Y, X) + \left[\widehat{Rm}(X, Y)Z + \widehat{Rm}(X, Z)Y\right]^{\xi}.
\]
\end{lemma}

\begin{proof}
First, since $g$ is symmetric, $C(X, Y, Z) = C(X, Z, Y)$:
\[
\begin{split}
C(X, Y, Z) - C(X, Z, Y) &= (\tilde{\nabla}_X g) (Y, Z) - (\tilde{\nabla}_X g) (Z, Y) \\
&= \tilde{\nabla}_X \left[g(Y, Z) - g(Z, Y)\right] \\
&\quad - g(\tilde{\nabla}_X Y, Z) - g(Y, \tilde{\nabla}_X Z) + g(\tilde{\nabla}_X Z, Y) + g(Z, \tilde{\nabla}_X Y) \\
&= 0.
\end{split}
\]

But if $\widehat{Rm} \ne 0$, $C$ may not be symmetric in the first two slots: using the structure equations,
\[
C(X, Y, Z) = C(Y, X, Z) + [\widehat{Rm}(X, Y)Z]^{\xi},
\]
where the superscript $\xi$ denotes the $\xi$ component in the splitting, $F^{\ast} TN \simeq TM \oplus \RR \xi$. Applying the structure equation to $C(Y, X, Z) = C(Y, Z, X)$, we obtain,
\begin{equation}
\label{eq:cubic_permutation1}
C(X, Y, Z) = C(Z, Y, X) + \left[\widehat{Rm}(Y, Z) X + \widehat{Rm}(X, Y) Z\right]^{\xi}.
\end{equation}
On the other hand, since $\diff_X \varphi = (\tilde{\nabla}_X - \nabla_X) \varphi = 0$ for any smooth function $\varphi$, applying $\diff_X$ to $g$ as a derivation gives,
\[
(\diff_X g) (Y, Z) = \diff_X (g(Y, Z)) - g(\diff_X Y, Z) - g(Y, \diff_X Z) = -g(\diff_X Y, Z) - g(Y, \diff_X Z).
\]
Now using $\nabla g \equiv 0$, gives
\begin{equation}
\label{eq:cubic_derivation_identity}
C(X, Y, Z) = (\tilde{\nabla}_X g) (Y, Z) = (\diff_X g) (Y, Z) + (\nabla_X g) (Y, Z) = -g(\diff_X Y, Z) - g(Y, \diff_X Z).
\end{equation}

Thus from \eqref{eq:cubic_permutation1}, \eqref{eq:cubic_derivation_identity} we obtain,
\begin{equation}
\label{eq:cubic_identity}
\begin{split} 
-g(\diff_X Y, Z) - g(Y, \diff_X Z) &= C(Z, Y, X) + \left[\widehat{Rm}(Y, Z) X + \widehat{Rm}(X, Y)Z\right]^\xi \\
&= -g(\diff_Z Y, X) - g(Y, \diff_Z X) + \left[\widehat{Rm}(Y, Z) X + \widehat{Rm}(X, Y)Z\right]^\xi
\end{split}
\end{equation}
Since $\diff_X Z = \diff_Z X$ this yields,
\[
-g(\diff_X Y, Z) = - g(\diff_Z Y, X) + \left[\widehat{Rm}(Y, Z) X + \widehat{Rm}(X, Y)Z\right]^\xi
\]
When applied to $g(\diff_X Z, Y)$ we obtain
\[
-g(\diff_Z X, Y) = - g(\diff_X Z, Y) = - g(\diff_Y Z, X) + \left[\widehat{Rm}(Z, Y) X + \widehat{Rm}(X, Z)Y\right]^\xi
\]
Finally, upon substitution of this identity into equation \eqref{eq:cubic_identity} we complete the proof with,
\begin{equation}
\begin{split}
C(X, Y, Z) &= -g(\diff_Z Y, X) - g(\diff_Z X, Y) + \left[\widehat{Rm}(Y, Z) X + \widehat{Rm}(X, Y)Z\right]^\xi \\
&= -g(\diff_Z Y, X) - g(\diff_Y Z, X) \\
&\quad + \left[\widehat{Rm}(Y, Z) X + \widehat{Rm}(X, Y)Z + \widehat{Rm}(Z, Y) X + \widehat{Rm}(X, Z)Y\right]^\xi \\
&= - 2g(\diff_Y Z, X) + \left[\widehat{Rm}(X, Y)Z + \widehat{Rm}(X, Z)Y\right]^{\xi}
\end{split}
\end{equation}
where we used the first anti-symmetry of $\widehat{Rm}$ in the last line.
\end{proof}

The second covariant derivatives of $ \xi$ is given by
\begin{align}
\hess\xi(X,Y)&=\D_Y\D_X\xi-\D_{\nabla_YX}\xi\\
&=-\D_YF_{\ast}(AX)+F_{\ast}(A\nabla_YX)\nonumber\\
&=-F_{\ast}(\tilde{\nabla}_{Y}AX)-g(A X,Y)\xi+F_{\ast}(A\nabla_YX)\nonumber\\
&=-F_{\ast}(\nabla_{Y}AX)-F_{\ast}(\diff_{Y}AX)-g(A X,Y)\xi+F_{\ast}(A\nabla_YX)\nonumber\\
&=-F_{\ast}((\nabla_Y A)X)-F_{\ast}(\diff_{Y}AX)-g(A X,Y)\xi.\nonumber
\end{align}

On the other hand, we see that if $\widehat{\Rm}$ does not vanish, then $\hess \xi$ is not symmetric:
\begin{align}
\widehat{Rm}(X,Y)\xi&=\D^2_{X,Y}\xi-\D^2_{Y,X}\xi\\
&=\hess\xi(X,Y)-\hess\xi(Y,X)-\D_{\D_YX-\D_XY}\xi-\D_{\nabla_XY-\nabla_YX}\xi\nonumber\\
&=\hess\xi(X,Y)-\hess\xi(Y,X).\nonumber
\end{align}

\section{evolution equations}

Perhaps the most important aspect of our approach is the following lemma, exhibiting a crucial property of our chosen parametrisation.

\begin{lemma}
\begin{align}
\partial_t\nu=0,
\end{align}
\end{lemma}

\begin{proof}
$\partial_t \nu$ is tangential, and
\begin{align*}
-\langle \partial_t\nu,X\rangle=&\langle \nu,\D_{\xi}X\rangle-\D_{\frac{\partial}{\partial t}}\langle \nu,X \rangle \\
=&\langle \nu,\D_{X}\xi\rangle\\
=&\langle \nu,-AX\rangle=0.
\end{align*}
\end{proof}

The next lemma contains other basic evolution equations.
\begin{lemma}
\begin{align}
\partial_t f = |p|(f\mathcal{H}+\Tr (\widehat{Rm}(\xi,\cdot,\cdot,\nu))),
\end{align}
\begin{align}
\partial_tg(X,Y)=&-\frac{p}{|p|}g(A(X),Y)+\frac{p}{|p|f}\langle \widehat{Rm}(\xi,Y)X,\nu\rangle\\
&-
|p|g(X,Y)(\mathcal{H}+\frac{1}{f}\Tr (\widehat{Rm}(\xi,\cdot,\cdot,\nu))).\nonumber
\end{align}
Here $\mathcal{H}$ is the mean curvature associated with $\xi;$ that is, $\mathcal{H}=g^{ij} A _{ij}=\sum  A _i^i.$
\end{lemma}

\begin{proof}
To calculate the evolution equation of $ f $, we need first to calculate the evolution equation of $\bar{g}$ and $h.$
\begin{align*}
\partial_t \bar{g}(X,Y)&=\D_{\frac{\partial}{\partial t}}\langle X,Y\rangle\\
&=\langle \D_X\xi,Y\rangle+\langle \D_Y\xi,X\rangle\\
&=-\bar{g}( AX,Y)-\bar{g}(X,AY).
\end{align*}

Using $h(X,Y)=\langle \D_{Y}X, \nu\rangle$ and $\partial_t\nu=0$ we calculate
\begin{align*}
\partial_t [h(X,Y)] &= \D_{\frac{\partial}{\partial t}}\langle \D_{Y}X,\nu\rangle\\
&=\frac{p}{|p|}\langle \D_{\xi}\D_{Y}X,\nu\rangle\\
&=\frac{p}{|p|}\langle \D_{Y}\D_{X}\xi,\nu\rangle+ \frac{p}{|p|}\langle \widehat{Rm}(\xi,Y)X,\nu\rangle\\
&=-\frac{p}{|p|}\langle \D_{Y}AX,\nu\rangle+ \frac{p}{|p|}\langle \widehat{Rm}(\xi,Y)X,\nu\rangle\\
&=\frac{p}{|p|}\langle AX,\D_{Y}\nu\rangle+ \frac{p}{|p|}\langle \widehat{Rm}(\xi,Y)X,\nu\rangle\\
&=-\frac{p}{|p|} f  g(A(X),Y)+\frac{p}{|p|}\langle \widehat{Rm}(\xi,Y)X,\nu\rangle.
\end{align*}
Therefore, using $\partial_t K=\frac{\partial K}{\partial h_{ij}}\partial_th_{ij}+\frac{\partial K}{\partial \bar{g}_{ij}}\partial_t\bar{g}_{ij}$ we get
\begin{align*}
\partial_tK=\frac{pK}{|p|}(\mathcal{H}+\Tr_h (\widehat{Rm}(\xi,\cdot,\cdot,\nu))).
\end{align*}
Thus
\begin{align}
 \partial_tf&=|p|(f\mathcal{H}+\Tr (\widehat{Rm}(\xi,\cdot,\cdot,\nu)))\nonumber\\
&= |p|(f\mathcal{H}+f\Tr (\widehat{Rm}(\nu,\cdot,\cdot,\nu))+\Tr (\widehat{Rm}(W,\cdot,\cdot,\nu))).
\end{align}

The evolution equation of $g$ follows from the evolution equations of $ f $ and $h:$
\begin{align*}
\partial_t [g(X,Y)] = &-\frac{p}{|p|}g(A(X),Y)+\frac{p}{|p|f}\langle \widehat{Rm}(\xi,Y)X,\nu\rangle\\
&-
|p|g(X,Y)(\mathcal{H}+\frac{1}{f}\Tr (\widehat{Rm}(\xi,\cdot,\cdot,\nu))).
\end{align*}

We also compute
\[
\begin{split}
(\D_{\xi} h) (X, Y) &= \partial_t[h(X, Y)] - h(\D_\xi X, Y) - h(X, \D_\xi Y) \\
&= -\frac{p}{|p|} f  g(A(X),Y)+\frac{p}{|p|}\langle \widehat{Rm}(\xi,Y)X,\nu\rangle + fg(AX, Y) + fg(X, AY) \\
&= \left(2-\frac{p}{|p|}\right) f  g(A(X),Y) + \frac{p}{|p|}\langle \widehat{Rm}(\xi,Y)X,\nu\rangle - f \left[\widehat{Rm}(X, Y)\xi\right]^{\xi},
\end{split}
\]
with the last line coming from the structure equation \eqref{eq:structure2} and the anti-symmetry of $\widehat{Rm}$ in the first two slots.

For the metric $g$, we have
\[
\begin{split}
(\D_{\xi} g) (X, Y) &= \partial_t[g(X, Y)] - g(\D_\xi X, Y) - g(X, \D_\xi Y) \\
&= -\frac{p}{|p|}g(A(X),Y)+\frac{p}{|p|f}\langle \widehat{Rm}(\xi,Y)X,\nu\rangle \\
&-
|p|g(X,Y)(\mathcal{H}+\frac{1}{f}\Tr (\widehat{Rm}(\xi,\cdot,\cdot,\nu))) \\
&\quad + g(AX, Y) + g(X, AY) \\
&= \left(2 -\frac{p}{|p|}\right) g(A(X),Y) -
|p|\mathcal{H} g(X,Y) \\
&\quad + \frac{p}{|p|f} \left[\langle \widehat{Rm}(\xi,Y)X,\nu\rangle -
\Tr (\widehat{Rm}(\xi,\cdot,\cdot,\nu))g(X,Y)\right] - \left[\widehat{Rm}(X, Y)\xi\right]^{\xi},
\end{split}
\]
\end{proof}

Next, we compute the evolution of $\xi = -\grad_h f + f \nu$. This is somewhat more involved than the previous lemma.

\begin{lemma}
\[
\D_{\xi} \xi = \left[1 -\frac{p}{|p|}\right] A(\grad_h f) +
|p|\mathcal{H}\xi - |p| f \grad_h\mathcal{H} + R
\]
where $R$ satisfies,
\[
\begin{split}
g(R - |p| \Tr (\widehat{Rm}(\xi,\cdot,\cdot,\nu))) \nu, X) &= |p|\frac{1}{f} g(\Tr \left(\widehat{Rm}(\xi, \cdot, \cdot, \nu)\right)\grad_h f, X) - |p| g(\grad_h \Tr \left(\widehat{Rm}(\xi, \cdot, \cdot, \nu)\right), X) \\
&\quad + \frac{p}{|p|f} \left[\langle \widehat{Rm}(\xi,\grad_h f)X,\nu\rangle -
\Tr (\widehat{Rm}(\xi,\cdot,\cdot,\nu))g(\grad_h f, X)\right] \\
&\quad  -\frac{p}{|p|} \left[\widehat{Rm}(X, \grad_h f)\xi\right]^{\xi}.
\end{split}
\]
\end{lemma}

\begin{proof}
\begin{equation}
\label{eq:dt_transverse_productrule}
\begin{split}
\D_{\xi}\xi &= -\D_{\xi} \left(\grad_h f\right) + (\partial_t f) \nu \\
&= -\D_{\xi}\left(\grad_h f\right) + |p|(f\mathcal{H}+\Tr (\widehat{Rm}(\xi,\cdot,\cdot,\nu))) \nu
\end{split}
\end{equation}
Now we use the defining equation for $\grad_h f$,
\[
X(\ln f) = \frac{1}{f} h(\grad_h f, X) = g(\grad_h f, X)
\]
We then compute
\[
\begin{split}
\D_{\xi} \left(X(\ln f)\right) &= \D_{\xi} \left[g(\grad_h f, X)\right] \\
&= (\D_{\xi} g) (\grad_h f, X) + g(\D_{\xi} \grad_h f, X) + g(\grad_h f,  \D_{\xi} X).
\end{split}
\]
Rearranging gives,
\begin{equation}
\label{eq:dt_transversegradient_structure}
\begin{split}
-g(\D_{\xi} \grad_h f, X) &= -\D_{\xi} \left(X(\ln f)\right) + (\D_{\xi} g) ((\grad_h f, X) + g(\grad_h,  \D_{\xi} X) \\
&= -X(\partial_t \ln f) + (\D_{\xi} g) ((\grad_h f, X) - g(\grad_h f,  A(X))
\end{split}
\end{equation}

The first term of equation \eqref{eq:dt_transversegradient_structure} is,
\[
\begin{split}
-X(\partial_t \ln f) &= -X\left[|p|(\mathcal{H} + \frac{1}{f} \Tr (\widehat{Rm}(\xi,\cdot,\cdot,\nu)))\right] \\
&= -|p| h(\grad_h \mathcal{H}, X) \\
&\quad + |p|\frac{1}{f^2} h(\Tr \left(\widehat{Rm}(\xi, \cdot, \cdot, \nu)\right)\grad_h f, X) - \frac{|p|}{f} h(\grad_h \Tr \left(\widehat{Rm}(\xi, \cdot, \cdot, \nu)\right), X) \\
&= -|p| fg(\grad_h \mathcal{H}, X) \\
&\quad + |p|\frac{1}{f} g(\Tr \left(\widehat{Rm}(\xi, \cdot, \cdot, \nu)\right)\grad_h f, X) - |p| g(\grad_h \Tr \left(\widehat{Rm}(\xi, \cdot, \cdot, \nu)\right), X),
\end{split}
\]
Using the symmetry of $\D_{\xi}$, the late two terms of equation \eqref{eq:dt_transversegradient_structure} are
\[
\begin{split}
(\D_{\xi} g) (X, \grad_h f) - g(\grad_h f, A(X)) &= \left(1 -\frac{p}{|p|}\right) g(\grad_h f, A(X)) -
|p|\mathcal{H} g(\grad_h f, X) \\
&\quad + \frac{p}{|p|f} \left[\langle \widehat{Rm}(\xi,\grad_h f)X,\nu\rangle -
\Tr (\widehat{Rm}(\xi,\cdot,\cdot,\nu))g(X,\grad_h f)\right] \\
&\quad - \left[\widehat{Rm}(X, \grad_h f)\xi\right]^{\xi} \\
&= g\left(\left[1 -\frac{p}{|p|}\right] A(\grad_h f) -
|p|\mathcal{H} \grad_h f, X\right) \\
&\quad + \frac{p}{|p|f} \left[\langle \widehat{Rm}(\xi,\grad_h f)X,\nu\rangle -
\Tr (\widehat{Rm}(\xi,\cdot,\cdot,\nu))g(\grad_h f, X)\right] \\
&\quad  -\frac{p}{|p|} \left[\widehat{Rm}(X, \grad_h f)\xi\right]^{\xi}
\end{split}
\]
upon using the structure equation \eqref{eq:structure2} to get the second line.

Therefore we may rewrite \eqref{eq:dt_transversegradient_structure} as
\begin{equation}
\label{eq:dt_transversegradient_structure2}
\begin{split}
-g(\D_{\xi} \grad_h \ln f, X) &= -|p| fg(\grad_h \mathcal{H}, X) \\
&\quad + |p|\frac{1}{f} g(\Tr \left(\widehat{Rm}(\xi, \cdot, \cdot, \nu)\right)\grad_h f, X) - |p| g(\grad_h \Tr \left(\widehat{Rm}(\xi, \cdot, \cdot, \nu)\right), X) \\
&\quad + g\left(\left[1 -\frac{p}{|p|}\right] A(\grad_h f) -
|p|\mathcal{H} \grad_h f, X\right) \\
&\quad + \frac{p}{|p|f} \left[\langle \widehat{Rm}(\xi,\grad_h f)X,\nu\rangle -
\Tr (\widehat{Rm}(\xi,\cdot,\cdot,\nu))g(\grad_h f, X)\right] \\
&\quad  -\frac{p}{|p|} \left[\widehat{Rm}(X, \grad_h f)\xi\right]^{\xi} \\
&= g\left(\left[1 -\frac{p}{|p|}\right] A(\grad_h f) -
|p|\mathcal{H} \grad_h f -|p| f \grad_h \mathcal{H}, X\right) \\
&\quad + |p|\frac{1}{f} g(\Tr \left(\widehat{Rm}(\xi, \cdot, \cdot, \nu)\right)\grad_h f, X) - |p| g(\grad_h \Tr \left(\widehat{Rm}(\xi, \cdot, \cdot, \nu)\right), X) \\
&\quad + \frac{p}{|p|f} \left[\langle \widehat{Rm}(\xi,\grad_h f)X,\nu\rangle -
\Tr (\widehat{Rm}(\xi,\cdot,\cdot,\nu))g(\grad_h f, X)\right] \\
&\quad  -\frac{p}{|p|} \left[\widehat{Rm}(X, \grad_h f)\xi\right]^{\xi}
\end{split}
\end{equation}

Since this is true for any $X$, this uniquely determines $-\D_{\xi} \grad_h f$, and upon substitution into equation \eqref{eq:dt_transverse_productrule2} we obtain
\begin{equation}
\label{eq:dt_transverse_productrule2}
\begin{split}
\D_{\xi}\xi &= -\D_{\xi}\left(\grad_h f\right) + |p|(f\mathcal{H} + \Tr (\widehat{Rm}(\xi,\cdot,\cdot,\nu))) \nu \\
&= \left[1 -\frac{p}{|p|}\right] A(\grad_h f) +
|p|\left[f \mathcal{H} - \mathcal{H} \grad_h f - f \grad_h\mathcal{H}\right] + R \\
&= \left[1 -\frac{p}{|p|}\right] A(\grad_h f) +
|p|\mathcal{H}\xi - |p| f \grad_h\mathcal{H} + R
\end{split}
\end{equation}
where $R$ satisfies,
\[
\begin{split}
g(R - |p| \Tr (\widehat{Rm}(\xi,\cdot,\cdot,\nu))) \nu, X) &= |p|\frac{1}{f} g(\Tr \left(\widehat{Rm}(\xi, \cdot, \cdot, \nu)\right)\grad_h f, X) - |p| g(\grad_h \Tr \left(\widehat{Rm}(\xi, \cdot, \cdot, \nu)\right), X) \\
&\quad + \frac{p}{|p|f} \left[\langle \widehat{Rm}(\xi,\grad_h f)X,\nu\rangle -
\Tr (\widehat{Rm}(\xi,\cdot,\cdot,\nu))g(\grad_h f, X)\right] \\
&\quad  -\frac{p}{|p|} \left[\widehat{Rm}(X, \grad_h f)\xi\right]^{\xi}.
\end{split}
\]
\end{proof}

For the purpose of obtaining a Harnack inequality, the main ingredient turns out to be the evolution of the affine mean curvature $\mathcal{H}$.

\begin{theorem} We have
\begin{align*}
\partial_t\mathcal{H}= |p| \Delta \mathcal{H}+\frac{p}{|p|}| A |^2+ |p|  \mathcal{H}^2+ |p|  g^{lk}\partial_l\mathcal{H} \Tr_{12}( C _{ijk}).
\end{align*}
\end{theorem}
\begin{proof}
We will first calculate the evolution equation of $ A _{ij}.$ Recall that $$\partial_i\xi=- A _i^k\partial_kF.$$ Thus
\begin{align*}
\frac{p}{|p|}\left(- p  g^{ij}\partial_i\mathcal{H}\partial_jF+ p  \mathcal{H}\xi\right)_i&=\partial_t\xi_i=-(\partial_t A _i^k)\partial_kF+\frac{p}{|p|} A _i^k A _j^k\partial_jF.
\end{align*}
We calculate
\begin{align*}
\partial_t A _i^k=\frac{p}{|p|}( A _i^m A _m^k+ p  \mathcal{H}_{;mi}g^{mk}+ p
g^{nm}\partial_n \mathcal{H} C _{mi}^k+ p  \mathcal{H} A _i^k).
\end{align*}
Therefore,
\begin{align*}
\partial_t A _{ij}&=\partial_t( A _i^kg_{kj})\\
&=\frac{p}{|p|}\left(\left( A _i^m A _m^k+ p  \mathcal{H}_{;mi}g^{mk}+ p
g^{nm}\partial_n \mathcal{H} C _{mi}^k+ p  \mathcal{H} A _i^k\right)g_{kj}- A _i^k
( A _{kj}+ p  \mathcal{H}g_{kj})\right).
\end{align*}
Rearranging terms gives
\[\partial_t  A _{ij}= |p|  (\mathcal{H}_{;ij}+  \partial_l\mathcal{H}  C _{ij}^l)\]
Therefore,
\begin{align*}
\partial_t\mathcal{H}&=|p|g^{ij}( \mathcal{H}_{;ij}+  \partial_l \mathcal{H} C ^l_{ij})+\frac{p}{|p|}( A ^{ij}+ p  \mathcal{H}g^{ij}) A _{ij}\\
&= |p| \Delta \mathcal{H}+\frac{p}{|p|}| A |^2+ |p|  \mathcal{H}^2+ |p|  g^{lk}\partial_l\mathcal{H} \Tr_{12}( C _{ijk}).
\end{align*}
\end{proof}

\section{The Harnack Inequality}

\begin{theorem}
\[\left\{
  \begin{array}{ll}
    \partial_t \left(\varphi(\nu)K^pt^{\frac{n p }{n p +1}}\right)> 0, & p>0; \\
    \partial_t \left(\varphi(\nu)K^pt^{\frac{n |p| }{n |p| -1}}\right)<0, & -\frac{1}{n}<p<0.
  \end{array}
\right.
\]
\end{theorem}
\begin{proof}
Suppose $p>0.$
By Lemma 1 we have
\begin{align*}
\partial_t \left(ft^{\frac{n p }{n p +1}}\right)&= p  t^{\frac{n p }{n p +1}-1}f\left(t\mathcal{H}+\frac{n}{n p +1}\right).
\end{align*}
Thus it suffices to show that $Q:=t\mathcal{H}+\frac{n}{n p +1}$ is always positive:
\begin{align*}
\partial_t Q&= p\Delta Q+ p  g^{lk}\partial_lQ \Tr_{12}( C _{ijk})+\mathcal{H}+t| A |^2+  tp \mathcal{H}^2\\
&\geq  p \Delta Q+ p  g^{lk}\partial_lQ \Tr_{12}( C _{ijk})+\mathcal{H}+t\frac{n p +1}{n} \mathcal{H}^2\\
&= p \Delta Q+ p  g^{lk}\partial_lQ \Tr_{12}( C _{ijk})+\frac{n p +1}{n}\mathcal{H}Q.
\end{align*}
Therefore, by the maximum principle, $Q$ is always positive.


Suppose $-\frac{1}{n}<p<0.$ We have
\begin{align*}
\partial_t \left(ft^{\frac{n |p| }{n |p| -1}}\right)&= |p|  t^{\frac{n |p| }{n |p| -1}-1}f\left(t\mathcal{H}+\frac{n}{n |p| -1}\right).
\end{align*}
Thus it suffices to show that $Q:=t\mathcal{H}+\frac{n}{n |p| -1}$ is always negative:
\begin{align*}
\partial_t Q&= |p| \Delta Q+ |p|  g^{lk}\partial_lQ \Tr_{12}( C _{ijk})+\mathcal{H}-t| A |^2+ |p|  t \mathcal{H}^2\\
&\leq  |p| \Delta Q+ |p|  g^{lk}\partial_lQ \Tr_{12}( C _{ijk})+\mathcal{H}+\frac{n |p| -1}{n}t \mathcal{H}^2\\
&= |p| \Delta Q+ |p|  g^{lk}\partial_lQ \Tr_{12}( C _{ijk})+\frac{n |p| -1}{n}\mathcal{H}Q.
\end{align*}
Therefore, by the maximum principle, $Q$ is negative positive.
\end{proof}

To get the usual Harnack estimate, we note if we define a time-dependent diffeomorphism $\varphi: M^n\to M^n$ by
\begin{align}
\partial_t\varphi^k=h^{kl}\partial_l K^{ p },
\end{align}
then $\bar{F}(x,t):=F(\varphi(x,t),t):M^{n}\to R^{n+1}$ satisfies
\begin{align}
\partial_t\bar{F}=\bar{K}^{ p }\nu,
\end{align}
where $\bar{K}(x,t):=K(\varphi(x,t),t)$. So we have
\begin{align}
\partial_t\bar{K}^{ p }-h^{kl}\partial_k\bar{K}^{ p }\partial_l\bar{K}^{ p }+\frac{n p }{(n p +1)t}\geq 0.
\end{align}

\bibliographystyle{amsplain}
\bibliography{Bibliography.bib}

\end{document}
