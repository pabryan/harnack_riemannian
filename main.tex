\documentclass{amsart}
\usepackage[ocgcolorlinks,linktoc=all]{hyperref}
\usepackage{cancel}
\hypersetup{citecolor=blue,linkcolor=red}
\newtheorem{theorem}{Theorem}
\newtheorem*{thmA}{Theorem}
\newtheorem*{thmB}{Theorem}
\newtheorem*{rem}{Remark}
\newtheorem*{thmmain}{Theorem}
\newtheorem{lemma}[theorem]{Lemma}
\newtheorem{proposition}[theorem]{Proposition}
\newtheorem*{propmain}{Proposition}
\newtheorem{corollary}[theorem]{Corollary}
\theoremstyle{definition}
\newtheorem{definition}[theorem]{Definition}
\newtheorem{example}[theorem]{Example}
\newtheorem{xca}[theorem]{Exercise}

\theoremstyle{remark}
\newtheorem{remark}[theorem]{Remark}

\newcommand{\abs}[1]{\lvert#1\rvert}
\numberwithin{equation}{section}

\newcommand{\blankbox}[2]{%
  \parbox{\columnwidth}{\centering
    \setlength{\fboxsep}{0pt}%
    \fbox{\raisebox{0pt}[#2]{\hspace{#1}}}%
  }%
}

\begin{document}

\title[]
 {}

\curraddr{}
\email{}
\date{\today}

\dedicatory{}
\subjclass[2010]{}
\keywords{}

\begin{abstract}

\end{abstract}

\maketitle

Let $(M^n,\bar{g})$ be an $n$-dimensional smooth manifold and let $F(\cdot,t)\colon M^n\hookrightarrow (N^{n+1},\hat{g})$ be a one-parameter family of smooth hypersurfaces immersions. We say that it is a solution of the $ p $-GCF if
\begin{align}
\partial_tF=\xi,
\end{align}
Here $\xi $ is a smooth transverse inward-pointing vector field defined by
\begin{align}
\xi:=-h^{kl}\partial_l K ^{ p }\partial_kF+ K ^{ p }\nu,
\end{align}
 where $ K $ is the Gauss curvature and $\nu$ is an inward unit normal vector. In what follows for simplicity, we set $ f = K ^{ p }.$
Write $\bar{\Gamma}_{ij}^k$ for the Christoffel symbols of the metric $\bar{g}.$ Recall that
\begin{align}\label{gauss equ}
\partial^2_{ij}F^{\alpha}=\bar{\Gamma}_{ij}^k\partial_kF^{\alpha}+h_{ij}\nu^{\alpha}-\hat{\Gamma}^{ \alpha }_{\beta\gamma}F^{\beta}_iF^{\gamma}_j,
\end{align}
where $h_{ij}$ the second fundamental form with respect to $\nu.$

A direct calculation shows that $\partial_i\xi$ is tangential. The second fundamental form with respect to $\xi$ is defined by
\begin{align}
\partial_i\xi:=-\mathcal{A}_i^j\partial_jF.
\end{align}
Define the non-degenerate metric
\begin{align}
g_{ij}:=\frac{h_{ij}}{ f }.
\end{align}
From the Gauss equation given by (\ref{gauss equ}) we obtain
\begin{align}
\partial^2_{ij}F^{\alpha}=g_{ij}\xi^{\alpha}+(\Gamma_{ij}^k+\mathcal{C}_{ij}^k)\partial_kF^{\alpha}-\hat{\Gamma}^{ \alpha }_{\beta\gamma}F^{\beta}_iF^{\gamma}_j,
\end{align}
where
\begin{align}
\Gamma_{ij}^k=\frac{1}{2}g^{kl}\left(\partial_ig_{jl}+\partial_jg_{il}-\partial_lg_{ij}\right),
\end{align}
and
\begin{align}
\mathcal{C}_{ij}^k:=&\bar{\Gamma}_{ij}^k+\frac{1}{2}\frac{h^{kl}}{ f }\left(h_{jl}\partial_i f +h_{il}\partial_j f +h_{ij}\partial_l f \right)-\frac{1}{2}\frac{h^{kl}}{ f }\partial_ih_{jl}\\
=&\frac{1}{2}\frac{h^{kl}}{ f }\left(h_{jl}\partial_i f +h_{il}\partial_j f +h_{ij}\partial_l f \right)-\frac{1}{2}h^{kl}h_{jl;i}.
\end{align}
 We call the tensor $\mathcal{C}_{ij}^k$ the cubic form.

For a general Riemannian manifold $(M,g)$, an immersion $F\colon M^n\hookrightarrow (N^{n+1},\hat{g})$ and a vector field $\xi$ in $TN$ along $M$, i.e.,
\[
\xi\colon M\rightarrow TN
\]
the second covariant derivatives of $\xi$ with respect to the metric $g$ are given by
\[
\xi^{ \alpha}_{;ij}=\xi^{ \alpha }_{,ij}-\Gamma_{ij}^k\xi^{ \alpha }_k+\hat{\Gamma}^{ \alpha }_{\beta\gamma}\xi^{\beta}_iF^{\gamma}_j.
\]
Hence the commutator is given by
\[
\xi_{;ij}^{ \alpha }-\xi_{;ji}^{ \alpha }=0.
\]


The second covariant derivatives of $F$ and $ \xi$ with respect to the metric $g_{ij}$ are given by
\begin{align}
F_{;ij}^{\alpha}=\partial^2_{ij}F^{\alpha}-\Gamma_{ij}^k\partial_kF^{\alpha}+\hat{\Gamma}^{ \alpha }_{\beta\gamma}F^{\beta}_iF^{\gamma}_j=g_{ij}\xi+\mathcal{C}_{ij}^k\partial_kF^{\alpha},
\end{align}
\begin{align}
\xi_{;ij}=-\mathcal{A}_{i;k}^j\partial_kF-\mathcal{A}_{ij}\xi^{\alpha}-\mathcal{A}_i^k\mathcal{C}_{kj}^l\partial_lF.
\end{align}
Since $\xi_{;ij}=\xi_{;ji},$ looking at the normal component of $\xi$ shows that $\mathcal{A}_{ij}=\mathcal{A}_{ji},$ and also looking at the tangential components yields the following Codazzi equations
\begin{align}
\mathcal{A}^k_{j;i}-\mathcal{A}^k_{i;j}&=\mathcal{A}_i^l\mathcal{C}_{lj}^k-\mathcal{A}^l_j\mathcal{C}_{li}^k\\
\mathcal{A}_{jk;i}-\mathcal{A}_{ik;j}&=\mathcal{A}_i^l\mathcal{C}_{ljk}-\mathcal{A}_j^l\mathcal{C}_{lik}.
\end{align}
Considering the third derivatives of $F$ and using the  commuting covariant derivatives of tensor imply that
\begin{align}
\mathcal{C}_{ikj}=\mathcal{C}_{ijk},
\end{align}
and thus the cubic form is totally symmetric in all three indices. %Furthermore, the cubic form satisfies
%\begin{align}
%\mathcal{C}_{ijl;k}-\mathcal{C}_{ikl;j}=\frac{1}{2}g_{ij}\mathcal{A}_{kl}-\frac{1}{2}g_{ik}\mathcal{A}_{jl}+\frac{1}{2}g_{lj}\mathcal{A}_{ki}-\frac{1}{2}g_{lk}\mathcal{A}_{ji}.
%\end{align}
\begin{lemma}
\begin{align}
\partial_t\nu=0,
\end{align}
\begin{align}
\partial_t f = p  \mathcal{H} f ,
\end{align}
\begin{align}
\partial_tg_{ij}=-\mathcal{A}_{ij}- p  \mathcal{H}g_{ij},
\end{align}
\begin{align}
\partial_t\xi=- p  g^{ij}\partial_i\mathcal{H}\partial_jF+ p  \mathcal{H}\xi,
\end{align}
%\begin{align}
%\partial_t \mathcal{C}_{ijm}=&- p  \mathcal{H}\mathcal{C}_{ijm}+\frac{ p }{2}(\partial_i\mathcal{H}g_{jm}+\partial_j\mathcal{H}g_{im}+\partial_m\mathcal{H}g_{ij})\\
%&-\frac{1}{2}(\mathcal{A}_{ij;m}-\mathcal{A}_m^l\mathcal{C}_{lij})-\frac{1}{2}\mathcal{A}_i^l\mathcal{C}_{ljm}-\frac{1}{2}\mathcal{A}_j^l\mathcal{C}_{lim}-\frac{1}{2}\mathcal{A}_m^l\mathcal{C}_{lij}.
%\end{align}
Here $\mathcal{H}$ is the mean curvature associated with $\xi;$ that is, $\mathcal{H}=g^{ij}\mathcal{A}_{ij}=\sum \mathcal{A}_i^i.$
\end{lemma}
\begin{proof}
Since $\partial_t \nu$ is tangential, we need only to calculate
\begin{align*}
\partial_t\nu&=\langle \partial_t\nu,\partial_iF\rangle \bar{g}^{ij}\partial_jF\\
&=-\langle \nu,\partial^2_{ti}F\rangle \bar{g}^{ij}\partial_jF\\
&=-\langle \nu,-\mathcal{A}_i^k\partial_kF\rangle \bar{g}^{ij}\partial_jF=0.
\end{align*}
To calculate the evolution equation of $ f $, we need first to calculate the evolution equation of $\det(\bar{g}_{ij})$ and $\det(h_{ij}).$
\begin{align*}
\partial_t \bar{g}_{ij}&=\partial_t \langle \partial_iF,\partial_jF\rangle\\
&=\langle \partial_{ti}^2F,\partial_jF\rangle+\langle \partial_{tj}^2F,\partial_iF\rangle\\
&=-\mathcal{A}_{i}^k\bar{g}_{kj}-\mathcal{A}_{j}^k\bar{g}_{ki}.
\end{align*}
Thus
\begin{align*}
\partial_t\det (\bar{g}_{ij})&=\det (\bar{g}_{ij})\bar{g}^{ij}\partial_t\bar{g}_{ij}\\
&=\det (\bar{g}_{ij})\bar{g}^{ij}
(\mathcal{A}_{i}^k\bar{g}_{kj}+\mathcal{A}_{j}^k\bar{g}_{ki})\\
&=2\det (\bar{g}_{ij})\mathcal{H}.
\end{align*}
On the other hand, by $h_{ij}=\langle \partial_{ij}^2F, \nu\rangle$ and $\partial_t\nu=0$ we have
\begin{align*}
\partial_t h_{ij}&=\langle\partial^3_{tij}F,\nu\rangle\\
&=\langle\partial^3_{ij}\xi,\nu\rangle\\
&=\langle \partial_i(-\mathcal{A}_j^k\partial_kF),\nu\rangle\\
&=-\mathcal{A}_i^kh_{kj}=- f  \mathcal{A}_{ij}.
\end{align*}
This in turn implies that $\partial_t\det(h_{ij})=-\det(h_{ij})\mathcal{H}.$
Thus
\begin{align*}
\partial_t K =\partial_t\left(\frac{\det(h_{ij})}{\det(\bar{g}_{ij})}\right)=\mathcal{H} K \Rightarrow \partial_t  f = p  \mathcal{H} f .
\end{align*}
The evolution equation of $g_{ij}$ follows from the evolution equations of $ f $ and $h_{ij}.$ The evolution equation of $\xi$ follows from the evolution equations of $g_{ij}, F,  f ,\nu.$ %Next we proceed to calculate the evolution equation of $\mathcal{C}_{ijk}.$ To do so, in a normal coordinates with  $\partial_kg_{ij}=\Gamma_{ij}^k=0$, using the evolution equation of $g_{ij}$ we calculate the evolution equation of $\partial_t\Gamma_{ij}^j:$
%\begin{align*}
%\partial_t\Gamma_{ij}^k=-\frac{ p }{2}\left(\partial_i\mathcal{H}\delta_j^k+\partial_j\mathcal{H}\delta_i^k-g^{kl}\partial_lg_{ij}\right)-\frac{1}{2}(\mathcal{A}^k_{j;i}+\mathcal{A}^k_{i;j}-g^{kl}\mathcal{A}_{ij;l}).
%\end{align*}
%Thus
%\begin{align*}
%\partial_tF_{;ij}&=(\partial_tF)_{;ij}-(\partial_t\Gamma_{ij}^k)\partial_kF\\
%&=\xi_{;ij}+\left\{\frac{ p }{2}\left(\partial_i\mathcal{H}\delta_j^k+\partial_j\mathcal{H}\delta_i^k-g^{kl}\partial_lg_{ij}\right)-\frac{1}{2}(\mathcal{A}^k_{j;i}+\mathcal{A}^k_{i;j}+g^{kl}\mathcal{A}_{ij;l})\right\}\partial_kF.
%\end{align*}
%We may also calculate $\partial_tF_{;ij}$ in a different way:
%\begin{align*}
%\partial_tF_{;ij}=&\partial_t(g_{ij}\xi+\mathcal{C}_{ij}^k\partial_kF)\\
%=&\left(- p  \mathcal{H}g_{ij}-\mathcal{A}_{ij}\right)\xi+g_{ij}(- p  g^{ij}\partial_i\mathcal{H}\partial_jF+ p  \mathcal{H}\xi)\\
%&+(\partial_t \mathcal{C}_{ij}^k)\partial_kF-\mathcal{C}_{ij}^l\mathcal{A}_l^k\partial_kF.
%\end{align*}
%Putting these together gives
%\begin{align*}
%\partial_t \mathcal{C}_{ij}^k=&-\frac{1}{2}\mathcal{A}^l_{i}\mathcal{C}_{lj}^k-\frac{1}{2}\mathcal{A}_j^l\mathcal{C}_{li}^k-\frac{1}{2}g^{kl}\mathcal{A}_{ij;l}+\mathcal{C}_{ij}^l\mathcal{A}_l^k+ p  g_{ij}g^{kl}\partial_l\mathcal{H}\\
%&+\frac{ p }{2}(\partial_i\mathcal{H}\delta_j^k+\partial_j\mathcal{H}\delta_i^k-g^{kl}\partial_lg_{ij}).
%\end{align*}
%To obtain the evolution equation of $\mathcal{C}_{ijk}$ note that
%$\partial_t\mathcal{C}_{ijl}=\partial_t(g_{kl}\mathcal{C}_{ij}^k).$
\end{proof}
%\begin{lemma} The following identities hold
%\begin{align*}
%\Delta \mathcal{C}_{ilm}=&\frac{1}{2}g_{im}\partial_l\mathcal{H}+\frac{1}{2}g_{lm}\partial_i\mathcal{H}+\frac{1}{2}g_{il}\partial_m\mathcal{H}\\
%&+\frac{n+2}{2}(\mathcal{A}_m^k\mathcal{C}_{kil}-\mathcal{A}_{il;m})+\frac{1}{2}\mathcal{H}\mathcal{C}_{ilm}\\
%&-2\mathcal{C}_{ik}^r\mathcal{C}_{mr}^p\mathcal{C}_{pl}^k+\mathcal{C}_{im}^r\mathcal{C}_{kr}^p\mathcal{C}_{pl}^k+\mathcal{C}_{lm}^r\mathcal{C}_{kr}^p\mathcal{C}_{pi}^k+\mathcal{C}_{jm}^r\mathcal{C}_{rp}^j\mathcal{C}_{il}^p\\
%&+\frac{ p (n+2)-1}{2}\left\{(\partial_i \ln K)_{;lm}-\frac{1}{2}g_{lm}\mathcal{A}_i^k\partial_k\ln K-\frac{1}{2}g_{im}\mathcal{A}_l^k\partial_k\ln K\right\}\\
%&-\frac{ p (n+2)-1}{2}\left\{\mathcal{C}_{il}^p\mathcal{C}_{mp}^k\partial_k\ln K+\frac{1}{2}\mathcal{A}_{mi}\partial_l\ln K+\frac{1}{2}\mathcal{A}_{ml}\partial_i\ln K\right\},
%\end{align*}
%and
%\begin{align*}
%\Delta \mathcal{A}_{ij}=&
%\end{align*}
%\end{lemma}
%\begin{proof}
Except in the case $ p =\frac{1}{n+2}$, $\mathcal{C}_{ijk}$ does not enjoy the luxury of being trace free with respect to each of its two indices:
\begin{align}
g^{ij}\mathcal{C}_{ij}^k=\frac{ p (n+2)-1}{2}g^{kl}\partial_l\ln K,\quad g^{ij}\mathcal{C}_{ijk}=\frac{ p (n+2)-1}{2}\partial_k\ln K.
\end{align}

%\end{proof}
\begin{theorem} We have
%\begin{align*}
%\partial_t|\mathcal{C}|^2=&
%\end{align*}
%and
\begin{align*}
\partial_t\mathcal{H}= p \Delta \mathcal{H}+|\mathcal{A}|^2+ p  \mathcal{H}^2+ p  g^{lk}\partial_l\mathcal{H} tr_{12}(\mathcal{C}_{ijk}).
\end{align*}
\end{theorem}
\begin{proof}
We will first calculate the evolution equation of $\mathcal{A}_{ij}.$ Recall that $$\partial_i\xi=-\mathcal{A}_i^k\partial_kF.$$ Thus
\begin{align*}
\left(- p  g^{ij}\partial_i\mathcal{H}\partial_jF+ p  \mathcal{H}\xi\right)_i&=\partial_t\xi_i=-(\partial_t\mathcal{A}_i^k)\partial_kF+\mathcal{A}_i^k\mathcal{A}_j^k\partial_jF.
\end{align*}
We calculate
\begin{align*}
\partial_t\mathcal{A}_i^k=\mathcal{A}_i^m\mathcal{A}_m^k+ p  \mathcal{H}_{;mi}g^{mk}+ p
g^{nm}\partial_n \mathcal{H}\mathcal{C}_{mi}^k+ p  \mathcal{H}\mathcal{A}_i^k.
\end{align*}
Therefore,
\begin{align*}
\partial_t\mathcal{A}_{ij}&=\partial_t(\mathcal{A}_i^kg_{kj})\\
&=\left(\mathcal{A}_i^m\mathcal{A}_m^k+ p  \mathcal{H}_{;mi}g^{mk}+ p
g^{nm}\partial_n \mathcal{H}\mathcal{C}_{mi}^k+ p  \mathcal{H}\mathcal{A}_i^k\right)g_{kj}-\mathcal{A}_i^k
(\mathcal{A}_{kj}+ p  \mathcal{H}g_{kj}).
\end{align*}
Rearranging terms gives
\[\partial_t \mathcal{A}_{ij}= p  \mathcal{H}_{;ij}+ p \partial_l\mathcal{H} \mathcal{C}_{ij}^l\]
Therefore,
\begin{align*}
\partial_t\mathcal{H}&=g^{ij}( p  \mathcal{H}_{;ij}+ p  \partial_l \mathcal{H}\mathcal{C}^l_{ij})+(\mathcal{A}^{ij}+ p  \mathcal{H}g^{ij})\mathcal{A}_{ij}\\
&= p \Delta \mathcal{H}+|\mathcal{A}|^2+ p  \mathcal{H}^2+ p  g^{lk}\partial_l\mathcal{H} tr_{12}(\mathcal{C}_{ijk}).
\end{align*}
\end{proof}
\begin{theorem}
\[\partial_t \left(K^{ p }t^{\frac{n p }{n p +1}}\right)\geq 0.\]
\end{theorem}
\begin{proof}
By Lemma 1 we have
\begin{align*}
\partial_t \left(K^{ p }t^{\frac{n p }{n p +1}}\right)&= p  t^{\frac{n p }{n p +1}-1}K^{ p }\left(t\mathcal{H}+\frac{n}{n p +1}\right).
\end{align*}
Thus it suffices to show that $X:=t\mathcal{H}+\frac{n}{n p +1}$ is always positive:
\begin{align*}
\partial_t X&= p \Delta X+ p  g^{lk}\partial_lX tr_{12}(\mathcal{C}_{ijk})+\mathcal{H}+t|\mathcal{A}|^2+ p  t \mathcal{H}^2\\
&\geq  p \Delta X+ p  g^{lk}\partial_lX tr_{12}(\mathcal{C}_{ijk})+\mathcal{H}+(\frac{n p +1}{n})t \mathcal{H}^2\\
&= p \Delta X+ p  g^{lk}\partial_lX tr_{12}(\mathcal{C}_{ijk})+\frac{(n p +1)}{n}\mathcal{H}X.
\end{align*}
\end{proof}
To get the usual Harnack estimate, we note if we define a time-dependent diffeomorphism $\theta: M^n\to M^n$ by
\begin{align}
\partial_t\theta^k=h^{kl}\partial_l K^{ p },
\end{align}
then $\bar{F}(x,t):=F(\theta(x,t),t):M^{n}\to R^{n+1}$ satisfies
\begin{align}
\partial_t\bar{F}=\bar{K}^{ p }\nu,
\end{align}
where $\bar{K}(x,t):=K(\theta(x,t),t)$. So we have
\begin{align}
\partial_t\bar{K}^{ p }-h^{kl}\partial_k\bar{K}^{ p }\partial_l\bar{K}^{ p }+\frac{n p }{(n p +1)t}\geq 0.
\end{align}
\end{document}

 We mention that the cubic form is totally symmetric in all three indices and is trace free with respect to each of  two indices.

Define \[P_{ij}:=\nabla_i\nabla_jS^{-1}+g_{ij}S^{-1},\]
where $g_{ij}$ and $\nabla$ are the metric and the connection of the unit sphere and $S^{-1}$ is the Gauss curvature considered as a function on the unit sphere. First, we would like to calculate the evolution equation of $P_{ij}.$
To do so, we recall that
\[\mathfrak{r}_{ij}=\nabla_i\nabla_jh+g_{ij}h\]
\[\partial_th=-S^{-1},\]
\[S=\det(\mathfrak{r}_{ij})/\det(g_{ij}),\]
\[\partial_t \mathfrak{r}_{ij}=-P_{ij},\]
\[\partial_t S^{-1}=S^{-1}\mathfrak{r}^{kl}P_{kl}.\]
Thus we may calculate
\begin{align*}
\partial_tP_{aj}=&\nabla_a\nabla_j(S^{-1}\mathfrak{r}^{kl}P_{kl})+
g_{aj}S^{-1}\mathfrak{r}^{kl}P_{kl}\\
=&\nabla_a(\mathfrak{r}^{kl}P_{kl}\nabla_jS^{-1}+S^{-1}P_{kl}\nabla_j\mathfrak{r}^{kl}+S^{-1}\mathfrak{r}^{kl}\nabla_jP_{kl})+g_{aj}S^{-1}\mathfrak{r}^{kl}P_{kl}\\
=&\mathfrak{r}^{kl}P_{kl}P_{aj}+\mathfrak{r}^{kl}\nabla_aP_{kl}\nabla_jS^{-1}-\mathfrak{r}^{kr}\mathfrak{r}^{ls}P_{kl}\nabla_jS^{-1}\nabla_a\mathfrak{r}_{rs}\\
&+S^{-1}P_{kl}\nabla_a\nabla_j\mathfrak{r}^{kl}-\mathfrak{r}^{kr}\mathfrak{r}^{ls}P_{kl}\nabla_j\mathfrak{r}_{rs}\nabla_aS^{-1}-S^{-1}\mathfrak{r}^{kr}\mathfrak{r}^{ls}\nabla_aP_{kl}\nabla_j\mathfrak{r}_{rs}\\
&+S^{-1}\mathfrak{r}^{kl}\nabla_a\nabla_jP_{kl}+\mathfrak{r}^{kl}\nabla_jP_{kl}\nabla_aS^{-1}-S^{-1}\mathfrak{r}^{kr}\mathfrak{r}^{ls}\nabla_jP_{kl}\nabla_a\mathfrak{r}_{rs}.
\end{align*}

\begin{align*}
\partial_t \mathfrak{r}^{ia}=\mathfrak{r}^{ic}\mathfrak{r}^{ab}P_{bc}.
\end{align*}
Next we would like to calculate the evolution equation
of
$Q_j^i:=\mathfrak{r}^{ia}P_{aj}.$
\begin{align*}
\partial_t Q_j^i=(Q^2)_i^j+\mathfrak{r}^{ia}\partial_tP_{aj}
\end{align*}
\begin{align*}
\\mathfrak{r}artial_tP_{aj}=&S^{-2}S^{kl}P_{kl}g_{aj}-6S^{-4}S^{kl}P_{kl}\nabla_aS\nabla_jS-2S^{-3}S^{kl}P_{kl}\nabla_a\nabla_jS\\
&-2S^{-3}S^{kl,mn}P_{kl}\nabla_a\mathfrak{r}_{mn}\nabla_jS-2S^{-3}S^{kl}\nabla_aP_{kl}\nabla_jS\\
&-2S^{-3}S^{kl,mn}\nabla_j\mathfrak{r}_{mn}P_{kl}\nabla_aS+S^{-2}S^{kl,mn,rs}P_{kl}\nabla_j\mathfrak{r}_{mn}\nabla_a\mathfrak{r}_{rs}\\
&+S^{-2}S^{kl,mn}\nabla_aP_{kl}\nabla_j\mathfrak{r}_{mn}+S^{-2}S^{kl,mn}P_{kl}\nabla_a\nabla_j\mathfrak{r}_{mn}\\
&-2S^{-3}S^{kl}\nabla_jP_{kl}\nabla_aS+S^{-2}S^{kl,mn}\nabla_jP_{kl}\nabla_a\mathfrak{r}_{mn}+S^{-2}S^{kl}\nabla_a\nabla_jP_{kl}.
\end{align*} 